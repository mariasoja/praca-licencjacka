\documentclass[12pt,a4paper]{report}
\linespread{1.5}
\usepackage[utf8]{inputenc}
\usepackage{polski}
\usepackage{amsmath}
\usepackage{amsfonts}
\usepackage{amssymb}
\usepackage{natbib} %bibtex

\newtheorem{definition}{Definicja}
\newtheorem{przyklad}{Przykład}
\newtheorem{tw}{Twierdzenie}
\newtheorem{dow}{Dowód}
\author{Maria Soja}
\title{Praca licencjacka}

\newcommand{\parauporzadkowana}[2]{\left\langle {#1}; {#2} \right\rangle}
\newcommand{\zbior}[1]{\left\lbrace {#1} \right\rbrace }

\newcommand{\tuple}[1]{\left\langle {#1} \right\rangle}
\begin{document}

\maketitle

\chapter{Wstep}

\chapter{Preliminaria}

%\begin{definition}[\citep{kraszewski2007wstkep}]
%Parę uporządkowaną elementów $a$ i $b$ oznaczamy przez $<a,b>$. Element $a$ nazywamy pierwszą współrzędną (poprzednikiem), zaś element b - drugą współrzędną (następnikiem) pary $<a,b>$. Podstawowa własność określająca parę uporządkowaną jest następująca:
%\begin{equation*}
%<a,b>=<c,d>\Leftrightarrow a=c \wedge b=d
%\end{equation*} 
%
%\end{definition}

W naszej pracy rozważane będą zagadnienia wymagające doprecyzowania wielu pojęć z zakresu dziedzin matematyki takich jak statystyka czy algebra. 


Zdefiniujmy parę uporządkowaną. Warto powiedzieć, że w wielu pracach definicje tam użyte są matematycznie niepoprawne.

\begin{definition}[\citep{kuratowski1966wstkep}]
Parą uporządkowana składającą się z poprzednika $a$ oraz następnika $b$ nazwiemy zbiór składający się z elementów poniżej opisanych i oznaczany:
\begin{equation*}
\parauporzadkowana{a}{b}=\zbior{\zbior{a}, \zbior{a,b}}.
\end{equation*}

\end{definition}

\begin{definition}[Iloczyn kartezjański\citep{kuratowski1966wstkep}]
Iloczynem kartezjańskim zbiorów $X$ i $Y$ nazywamy zbiór wszystkich par uporządkowanych $\parauporzadkowana{x}{y}$, gdzie $x \in X$ i $y \in Y$. Zbiór ten oznaczamy przez $ X \times Y$; zatem
\begin{equation*}
\parauporzadkowana{x}{y} \in (X \times Y) \equiv (x \in X) \land (y \in Y).
\end{equation*} 

\end{definition}

\begin{definition}[\citep{kraszewski2007wstkep}]
Dane są dwa zbiory $X$ i $Y$. Relacją (dwuargumentową) między elementami zbioru $X$, a elementami zbioru $Y$ nazywamy dowolny podzbiór iloczynu kartezjańskiego $X \times Y$.
\end{definition}
Chcąc zapisać, że pewien element $x$ jest w relacji $R$ z pewnym elementem $y$ piszemy
$$
xRy.
$$

Najpowszechniej stosowaną relacją jest funkcja.

\begin{definition}[Funkcja\citep{kuratowski1966wstkep}]
Niech dane bedą dwa zbiory $X$ i $Y$. Przez funkcję, której argumenty przebiegają zbiór $X$, wartości zaś należą do zbioru $Y$, rozumiemy każdy podzbiór $f$ iloczynu kartezjańskiego $X \times Y$ o tej własności, że dla każdego $x \in X$ istnieje jeden i tylko jeden $y$ taki, że $\parauporzadkowana{x}{y} \in f$. 

\end{definition}
\begin{definition}[Superpozycja funkcji\citep{kuratowski1966wstkep}]
Niech dane będą trzy zbiory $X$, $Y$ i $Z$ oraz dwie funkcje $f:X\to Y$, oraz $g:Y\to Z$. Funkcje te wyznaczają trzecia funkcję złożoną $h:X\to Z$ (nazywaną superpozycją funkcji $f$ i $g$) określoną przez warunek
\begin{equation*}
\forall_{x \in X} \quad h(x)=g(f(x)).
\end{equation*}
\end{definition}

\begin{definition}[\citep{kraszewski2007wstkep}]
Niech $f:X \to Y$.
\begin{enumerate}
\item
Mówimy, że funkcja $f$ jest różnowartościowa i piszemy $f:X\xrightarrow{1-1} Y$, jeśli różnym argumentom przyporządkowuje ona różne wartości, czyli
\begin{equation*}
\forall{x_{1},x_{2}}\in X \quad x_{1}\ne x_{2} \implies f(x_{1})\ne f(x_{2}).
\end{equation*}
Funkcję taką nazywamy też injekcją lub mówimy, że jest $1-1$. Potocznie mówi się, że funkcja różnowartościowa nie skleja argumentów.
\item
Mówimy, że funkcja $f$ jest na i piszemy $f:X\xrightarrow{na}Y$, jeśli każdy element jej przeciwdziedziny jest wartością funkcji dla pewnego jej argumentu, czyli
\begin{equation*}
\forall_{y\in Y} \exists_{x\in X} \quad y=f(x).
\end{equation*}
Funkcję taką nazywamy też surjekcją.
\item
Jeśli funkcja $f$ jest różnowartościowa i na, nazywamy ją wzajemnie jednoznaczną i piszemy $f:X\xrightarrow[na]{1-1} Y$. O takiej funkcji mówimy też, że jest bijekcją.
\end{enumerate}
\end{definition}

\begin{definition}[\citep{kraszewski2007wstkep}]
Dany jest zbiór $X$. Jeśli $R$ jest relacją między elementami zbioru $X$ a elementami zbioru $X$ (czyli $R\subseteq X \times X$), to mówimy, że $R$ jest relacją na zbiorze $X$ (można też mówić o relacji w zbiorze $X$).
\end{definition}

\begin{definition}[\citep{kraszewski2007wstkep}]
Jeśli $R\subseteq X\times Y$ i $S\subseteq Y \times Z$ są relacjami, to złożeniem relacji $R$ z relacją $S$ nazywamy relację $S\circ R \subseteq X \times Z$ określoną warunkiem
\begin{equation*}
x(S\circ R)z \iff \exists_{y \in Y} (xRy \land ySz).
\end{equation*}
\end{definition}
\begin{definition}
Jeśli $R \subseteq X \times Z$ jest relacją, to relacją do niej odwrotną nazywamy relację $R^{-1} \subseteq Y \times X$ określoną warunkiem 
\begin{equation*}
R^{-1} := \zbior{\parauporzadkowana{y}{x} \in Y \times X; xRy}.
\end{equation*}
\end{definition}

\begin{definition}[\citep{kraszewski2007wstkep}]
Niech $R$ będzie relacją na niepustym zbiorze $X$. Mówimy, że:
\begin{enumerate}
\item
$R$ jest zwrotna $\iff$ $(\forall_{x\in X}) xRx$.
\item
$R$ jest przeciwzwrotna $\iff$ $(\forall_{x\in X})$ $\neg xRx$.
\item
$R$ jest przechodnia $\iff$ $(\forall_{x,y,z\in X})$ $(xRy\land yRz  \implies xRz)$.
\item
$R$ jest symetryczna $\iff$ $(\forall_{x,y\in X})(xRy \implies yRx)$. 
\item
$R$ jest słabo antysymetryczna $\iff$ $(\forall_{x,y\in X})( xRy \land yRx \implies x=y$).
\item
$R$ jest silnie antysymetryczna $\iff$ $(\forall_{x,y\in X})(xRy \implies \neg yRx$).
\item
$R$ jest spójna $\iff$ $(\forall_{x,y\in X})(xRy \lor yRx)$.

\end{enumerate}

\begin{definition}[Działanie]
Niech $X$ będzie dowolnym niepustym zbiorem. Powiemy, że $\ast$ jest działaniem w zbiorze $X$ jeśli $\ast: X \times X \to X$.
\end{definition}

\end{definition}
\begin{definition}[\citep{jedrzejewski2011algebra}]
Działanie $\ast$ w zbiorze $X$ nazywamy działaniem łącznym, jeśli spełniony jest warunek 
\begin{equation*}
(a\ast b)\ast c = a\ast(b\ast c)
\end{equation*} 
dla każdych elementów $a,b,c$ ze zbioru $X$.
\end{definition}
\begin{definition}[\citep{jedrzejewski2011algebra}]
Działanie $\ast$ w zbiorze $X$ nazywamy działaniem przemiennym, jeśli spełnia ono warunek
\begin{equation*}
a\ast b=b\ast a
\end{equation*}
dla każdych elementów $a,b$ ze zbioru $X$.
\end{definition}
\begin{definition}[\citep{jedrzejewski2011algebra}]
Element $e$, należący do zbioru $X$, nazywamy elementem prawostronnie neutralnym działania $\ast$ w zbiorze $X$, jeśli
\begin{equation*}
x\ast e = x
\end{equation*} 
dla każdego elementu $x$ ze zbioru $X$.
\end{definition}
\begin{definition}[\citep{jedrzejewski2011algebra}]
Element $e$, należący do zbioru $X$, nazywamy elementem lewostronnie neutralnym działania $\ast$ w zbiorze $X$, jeśli
\begin{equation*}
e\ast x = x
\end{equation*} 
dla każdego elementu $x$ ze zbioru $X$.
\end{definition}
\begin{definition}[\citep{jedrzejewski2011algebra}]
Element $e$ nazywamy elementem neutralnym w zbiorze $X$, jeśli jest on lewostronnie i prawostronnie neutralny, czyli
\begin{equation*}
e\ast x = x\ast e = x
\end{equation*}
dla każdego elementu $x$ ze zbioru $X$.
\end{definition}
\begin{definition}[Struktury algebraicznej \citep{jedrzejewski2011algebra}]
Strukturę algebraiczną (systemem algebraicznym, czasem algebrą) nazywamy niepusty zbiór wraz z pewną liczbą działań wewnętrznych i pewną liczbą działań zewnętrznych w tym zbiorze.
\end{definition}
\begin{definition}[Grupy \citep{jedrzejewski2011algebra}]
Strukturę algebraiczną ($\mathcal{G},\ast$) nazywamy grupą, jeśli spełnione są następujące warunki:
\begin{equation*}
\forall_{a\in \mathcal{G}} \forall_{b\in \mathcal{G}} \forall_{c\in \mathcal{G}} ((a\ast b)\ast c=a\ast (b\ast c)),
\end{equation*}
\begin{equation*}
\exists_{e\in \mathcal{G}} \forall_{a\in \mathcal{G}}(e\ast a=a=a\ast e),
\end{equation*}
\begin{equation*}
\forall_{a \in \mathcal{G}} \exists_{\overline{a}\in\mathcal{G}}(\overline{a}\ast a=e=a\ast \overline{a}).
\end{equation*}
\end{definition}
\begin{definition}[\citep{jedrzejewski2011algebra}]
Grupą przemienną (abelową) nazywamy taką grupę, w której działanie jest przemienne.
\end{definition}
%\begin{definition}[Funkcja kardynalna\citep{engelking1989topologia}]
%Funkcja kardynalna nazywamy ...
%\end{definition}


\chapter{Teoria}

Standardowo w analizie pomiaru, wynik będziemy przedstawiać jako funkcję o wartości liczbowej. Pomiary masy w funtach będą reprezentowane przez funkcję, która może być oznaczona jako "$Ib$";  ta funkcja jest skolerowana z każdym obiektem $x$ który można zważyć wartością liczbową, $Ib(x)$ - waga $x$ w funtach. Jakąkolwiek funkcję tego rodzaju nazywamy \textit{numerical assignement}. Choć w języku polskim nie jest dostępne utarte tłumaczenie tego pojęcia rozumiemy je jako formę przypisania liczby do danego obiektu. Na przykład przy pomiarze wagi możemy użyć rożnych jednostek. Jest to formalnie przedstawione  przez klasę przypisań liczbowych (funkcja funta, funkcja uncji, funkcja tona, itp.). Mówiąc ogólnie o pomiarze, przyjmiemy jako podstawowe pojecie klasę przypisań liczbowych. Drugim podstawowym pojęciem jest dopuszczalne przekształcenie czyli funkcja odwzorowującą wartości jednego numerycznego przypisania danej klasy na wartości innego przypisania numerycznego. Te dwa pojęcia składają się na numerical assignment system. Zdefiniujmy formalnie czym jest numerical assignment system.
\begin{definition}[Numerical assignment system (NAS)]
Numerical assignment system to uporządkowany układ $\tuple{A, M, a, \Phi}$ spełniający poniższe warunki:
\begin{enumerate}
\item
$A$ jest niepustym zbiorem, $a$ jest zbiorem liczb rzeczywistych, $M$ jest zbiorem funkcji przekształcających $A$ w $a$, i $\Phi$ jest klasą funkcji przekształcających $a$ w siebie.  
\item
Dla wszystkich $m$ z $M$ i $\phi$ z $\Phi$, 
$$
\phi \circ m \in M.
$$ 
\item
$\Phi$ zawiera tożsamościową transformację (identyczność) $a$ w siebie i dla wszystkich $\phi_{1}$ i $\phi_{2}$ z $\Phi$, 
$$
\phi_{1} \circ \phi_{2} \in \Phi.
$$

\end{enumerate}

\end{definition}
Warunek 2. powyższej definicji jest po prostu wymogiem na to aby każde dopuszczalne przekształcenie  przenosiło dowolne numeryczne przypisanie do innego numerycznego przypisania tego samego systemu. Warunek 3. nie jest konieczny, ale może być wymuszony, ponieważ tożsamościowe transformacje zawsze związane są z numerycznym przypisaniem na numeryczne przypisanie, jeżeli dwie transformacje maja tę samą właściwość to muszą tez mieć ich składniki. Można zauważyć, że pewne rzeczy przyjmowane za pewnik dotyczące pomiarów i dopuszczalnych przekształceń nie są zakładane  powyższej definicji. Po pierwsze, nie zakładaliśmy, że dopuszczalne przekształcenia stanowią grupę,która wymaga dodatkowych, dla każdego $\phi$ z $\Phi$, że $\Phi$ zawiera odwrotność $\phi$, a to z kolei wymagałoby żeby dodatkowo wszystkie $\phi$ z $\Phi$ były odwzorowaniem różnowartościowym $a$ w siebie. Nie zakładaliśmy ze numeryczne przypisania z $M$ odwzorowują $A$ na $a$ to znaczy, że pomiary przyjmują wszystkie możliwe wartości liczbowe. Wreszcie co może wydawać najpoważniejszym pominięciem, nie zakładaliśmy odwrotności Warunku 2. w definicji, to znaczy, że każde numeryczne przypisanie na ten system może być przeniesione na inny za pomocą dopuszczalnego przekształcenia. Omawiając podstawowe systemy pomiarów powinniśmy pokazać dlaczego nie zakładamy mocniejszych założeń. Dotyczy to empirycznej podstawy pojęć numerycznego przypisania odpowiadającego rodzajowi pomiaru i ego klasie dopuszczalnych przekształceń. Jeżeli nasze założenia o początku tych idei jest poprawne, to powyższe założenia ogólnie nie maja miejsca i dlatego wskazane jest zacząć od słabszych założeń ale lepiej uzasadnionych. Zasadniczy powód mocniejszych założeń wydaje się być prostota matematyczna. Łatwo zauważyć, że jeżeli te założenia nie są spełnione 
wówczas nie następują pewne wyniki dotyczące warunków dla niezmienności w ramach przekształceń odpowiadających danym typom skal. Wskazane jest zatem sformułowanie zestawu minimalnych dodatkowych założeń, które są wystarczające oby otrzymać pożądane konsekwencje, ale które mogą być uzasadnione dla wielu różnych rzeczywistych pomiarów w nauce.
\begin{definition}
Numerical assignment system $U=\tuple{A, M, a, \Phi}$ jest regularny wtedy i tylko wtedy gdy spełnione są poniższe warunki:
\begin{enumerate}
\item
Elementy z $M$ i $\Phi$ są różnowartościowe.
\item
Istnieje $m_{0}$ w $M$, które odwzorowuje $A$ w $a$ i które generuje $M$ w tym sensie, że dla każdego $m$ z $M$ istnieje $\phi$ z $\Phi$ takie, że $\phi \circ m_{0}=m$.
\item
Dla wszystkich skończonych $n$ i elementów $\alpha_{1},\dots,\alpha_{n}$ z $a$ i elementów $\phi$ z $\Phi$, istnieje funkcja $\Psi$ w $\Phi$, która odwzorowuje $a$ w $a$, tak że $\Psi(\Phi(\alpha_{i}))=\alpha_{i}$, $i=1,\ldots,n$.
\end{enumerate}
\end{definition}
\section{Typy skal}
Pojęcie Stevensa na temat typów skal jest jednym z najważniejszych założeń w jego teorii, jednakże inni autorzy nie zawsze używają go o tym samym znaczeniu, co może oznaczać, że jest pewna swoboda sposobów w której może być ono bardziej precyzyjne. Na przykład skala ilorazowa została opisana jako skala której transformacje tworzą grupę prawdopodobieństwa (mnożenie przed dodatnią stałą), ale nie określono jaka jest dziedzina tych klas transformacji.  
\begin{definition}
Niech $U=\tuple{A, M, a, \Phi}$ będzie NAS. Wtedy:
\begin{enumerate}
\item
$U$ jest skalą nominalną wtedy i tylko wtedy gdy $\Phi$ jest zbiorem wszystkich funkcji odwzorowujących $a$ w siebie.
\item
$U$ jest skalą porządkową wtedy i tylko wtedy gdy $\Phi$ jest zbiorem ściśle monotonicznych rosnących odwzorowań $a$ w siebie.
\item
$U$ jest skalą interwałową wtedy i tylko wtedy gdy $a$ jest zbiorem wszystkich liczb rzeczywistych i $\Phi$ jest zbiorem wszystkich funkcji $\phi$, takich że dla dow. $\beta,\gamma$ gdzie $\beta>0$
\begin{equation*}
\phi(\alpha)=\beta\alpha+\gamma
\end{equation*}
dla wszystkich $\alpha$ z $a$.
\item
$U$ jest skalą ilorazową wtedy i tylko wtedy gdy $a$ jest zbiorem liczb dodatnich rzeczywistych i $\Phi$ jest zbiorem wszystkich przekształceń takich że dla dow. $\beta>0$
\begin{equation*}
\phi(\alpha)=\beta\alpha
\end{equation*}
dla wszystkich $\alpha$ z $a$.
\end{enumerate}
\end{definition}

%TODO przykłady
\begin{przyklad}
Niech $A$ będzie zbiorem nazw dni tygodnia czyli  $$
A=\zbior{\textrm{Poniedziałek, wtorek, Środa, Czwartek, Piątek, Sobota, Niedziela}}
$$
oraz niech $a=\zbior{0,1,2,3,4,5,6}$, $M$ będzie zbiorem wszystkich przekształceń $A \to a$ i $\Phi$ będzie klasą funkcji przekształcających $a$ w siebie wtedy taki układ $U$ jest skalą nominalną.
\end{przyklad}
Zauważmy, że w powyższej skali nie jesteśmy w stanie uporządkować naszych danych np. od najlepszej do najgorszej. 
\begin{przyklad}
Niech $A=\zbior{\textrm{bardzo niski, niski, średni, wysoki, bardzo wysoki}}$, $a=\zbior{1,2,3,4,5}$, $M$ będzie zbiorem wszystkich przekształceń $A \to a$ i $\Phi$ będzie klasą funkcji przekształcających $a$ w siebie wtedy taki układ $U$ jest skalą porządkową.
\end{przyklad}
Natomiast w skali porządkowej już jesteśmy w stanie uporządkować nasze dane lecz nie możemy wykonywać żadnych działań.
\begin{przyklad}
Przykładem skali interwałowej jest układ $U$ gdzie zmienna jest rokiem urodzenia czyli niech $A=\zbior{1980,1995,1997,2000,2009}$, $a=\zbior{1,2,3,4,5}$, $M$ będzie zbiorem wszystkich przekształceń $A \to a$ i $\Phi$ będzie klasą funkcji przekształcających $a$ w siebie.
\end{przyklad}
W tej skali już na przykład można określić o ile lat ktoś jest straszy od innej osoby 
\begin{przyklad}
Natomiast przykładem skali ilorazowej jest układ $U$ gdzie zmienną jest wysokość mierzona w centymetrach czyli niech $A=\zbior{150,155,160,165,170,175}$, $a=\zbior{0,1,2,3,4,5}$, $M$ będzie zbiorem wszystkich przekształceń $A \to a$ i $\Phi$ będzie klasą funkcji przekształcających $a$ w siebie.
\end{przyklad}
W skali ilorazowej możliwe jest dokonywanie wszystkich operacji matematycznych.
\begin{tw}
Dla dowolnego $m\in M$ (należącego do NAS) $$
\forall_{y,z \in A} y=z \iff m(y)=m(z).
$$
\end{tw}
\begin{dow}
Rozważmy $y,z \in A$ i niech $m \in M$. Wtedy
\begin{equation*}
\exists_{\phi \in \Phi} m=\phi \circ \overline{m}.
\end{equation*}
Jeśli $y=z$ to $\overline{m}(y)=\overline{m}(z)$. $\phi$ jest funkcją zatem 
\begin{equation*}
\phi \circ \overline{m}(y)=\phi \circ \overline{m}(z) \iff m(y)=m(z).
\end{equation*}
Jeśli $y\ne z$ to $\overline{m}(y)\ne \overline{m}(z)$. $\phi$ jest różnowartościowa zatem
\begin{equation*}
\phi \circ \overline{m}(y) \ne \phi \circ \overline{m}(z) \iff m(y) \ne m(z)
\end{equation*}
Przypuśćmy, że $m(y)=m(z) \land y \ne z$. Wobec powyższego sprzeczność. Zatem
\begin{equation*}
\forall_{y,z \in A} y=z \iff m(y)=m(z).
\end{equation*}
\end{dow}
\begin{tw}
W skali nominalnej nie ma porządku.
\end{tw}
\begin{dow}
Przypuśćmy, że $\le$ jest porządkiem w $A$. Wtedy
\begin{equation*}
\forall_{m\in M} y \le z \implies m(y) \le m(z).
\end{equation*}
Pokażemy sprzeczność. $y\le z \impies \overline{m}(y)\le \overline{m}(z)$ oraz $y\ne z$. Rozważmy $\phi :a \to a$ $\forall_{v \in a} \phi(v) = \sigma -v$. Zauważmy, że $\phi$ jest funkcją liniową która nie jest stała, zatem jest różnowartościowa. Wtedy $y \le z \implies \overline{m}(y) \le \overline{m}(z)$. Skoro $\forall_{m \in M} y\le z \implies m(y) \le m(z)$ to 
\begin{equation*}
\phi \circ \overline{m}(y) \le \phi \circ \overline{m}(z) \iff \sigma -\overline{m}(y) \le \sigma - \overline{m}(z)\iff -\overline{m}(y) \le -\overline{m}(z) \iff \overline{m}(y) \ge \overline{m}(z)
\end{equation*}
\begin{equation*}
\overline{m}(y) \le \overline{m}(z) \land \overline{m}(y) \ge \overline{m}(z)
\end{equation*}
zatem $y=z$ co jest sprzeczne z założeniem.

\end{dow}




\section{Operacje statystyczne}
Poprzednie definicje dostarczają niezbędnych podstaw do precyzyjnych definicji różnych rodzajów niezmienności operacji statystycznych lub obliczeń stosowanych do pomiarów. Sformułujemy jeszcze jedną wstępną koncepcje to znaczy operacje statystyczną  (bardziej ogólnie, operację matematyczną) pomiarów. Liczba różnych rodzajów działań matematycznych  do których stosuje się pojęcie niezmienności jest dość duża i należałoby sformułować bardzo ogólną definicje działań matematycznych która by uwzględniała wszystkie poszczególne przypadki jednakże byłaby zbyt rozbudowana. Skupmy się na specjalnej klasie działań matematycznych i statystycznych, w których wynik jest obliczany ze skończonej liczby pomiarów (wielkość nie musi być ustalona) w pewien jednakowy sposób. Zawiera to w szczególnych przypadkach obliczenie średniej, mediany, odchylenia standardowego ze skończonej próbki, a także bardziej elementarne działania matematyczne takie jak na przykład dodawanie i odejmowanie. Nie zawiera działań stosowanych do nieskończonych populacji. Wszystkie standardowe operacje statystyczne mogą być przedstawione jako specjalne przypadki uogólnionej funkcji rzeczywistej. W celu zdefiniowania tego pojęcia powinniśmy wprowadzić następujące pojęcia pomocnicze. Jeżeli $A$ jest dowolnym zbiorem to $\overline{A}$ jest jego domknięciem stworzonym ze skończonych ciągów, mianowicie $\overline{A}$ składa się z $A$, wraz ze skończonymi ciągami z $A$ i skończonymi ciągami tych elementów i tak dalej. Jeżeli $m$ jest dowolną funkcją której dziedziną jest $A$, to można ją rozszerzyć na funkcję nad $\overline{A}$ w następujący sposób:
\begin{enumerate}
\item
Dla wszystkich $x$ z $A$, $m(x)$ jest zdefiniowana jak wcześniej.
\item
Dla dowolnego ciągu $x_{1},\dots,x_{n}$ z elementów z $\overline{A}$, $m( x_{1},\dots,x_{n})$ jest zdefiniowane jako ciąg $m(x_{1}),\dots,m(x_{n})$
\end{enumerate} 
Więc jeżeli $x$ jest dowolnym ciągiem, $m(x)$ jest po prostu odpowiadającym ciągiem wartości. Używając tego pojęcia możemy zdefiniować uogólnioną funkcję rzeczywistą której dziedzina jest podzbiorem $\overline{Re}$, gdzie $\overline{Re}$ jest zbiorem wszystkich liczb rzeczywistych. Większość konkretnych funkcji funkcji, które chcemy uwzględnić mają zdecydowanie bardziej ograniczone dziedziny, ale pożądane jest rozważenie wszystkich funkcji tej klasy razem. Przykładem takiej funkcji jest wzięcie średniej  liczb dowolnego takiego ciągu. Funkcja której dziedziną są wszystkie uporządkowane pary liczb rzeczywistych jest binarne działanie dodawania. Działanie które znajduje różnicę dwóch średnich może być rozumiane jako działanie posiadające dziedzinę składającą się klasy wszystkich uporządkowanych par skończonego ciągu liczb rzeczywistych. Uogólnione działanie na pomiarach jest intuicyjnie wynikiem zastosowania uogólnionej funkcji rzeczywistej  do liczb przypisanych przez liczbowe przypisanie. Zatem uogólnione operacje na pomiarach są generowane przez odpowiednie uogólnione funkcje rzeczywiste. Formalne definicje tych dwóch pojęć są następujące.
\begin{definition}
Uogólniona funkcja rzeczywista jest to dowolna funkcja której dziedziną jest podzbiór $\overline{Re}$, gdzie $Re$ jest zbiorem liczb rzeczywistych.
\end{definition}
\begin{definition}
Jeżeli $\mathcal{F}$ jest uogólnioną funkcją rzeczywistą z dziedziną $\mathcal{D}\subseteq\overline{Re}$, wtedy uogólnionym działaniem na pomiarach odpowiadające $\mathcal{F}$ jest funkcja $F$ której dziedzina $D$ jest zbiorem uporządkowanych par $m;x$ takich że
\begin{enumerate}
\item
$m$ jest funkcją o wartościach rzeczywistych na pewnej dziedzinie $A$;
\item
$x$ jest elementem z $\overline{A}$ takim, że $m(x)$ należy do $\mathcal{D}$;
\item
dla każdego $m;x$ z $D$, $F(m;x)=\mathcal{F}(m(x))$.
\end{enumerate}

\end{definition}
Użycie średnika w definicji dla pary $m;x$ nie ma znaczenia matematycznego. Należy zauważyć, że w definicji uogólnionej funkcji rzeczywistej , nie określono zakresu dla takich funkcji, zatem zakresy uogólnionych funkcji rzeczywistych może być zbiory arbitralne.W dalszej części będziemy się zajmować głównie funkcjami, których zakresy są zestawami liczb rzeczywistych. 
\chapter{Praktyka}

\chapter{Podsumowania}

\bibliographystyle{plain}
\bibliography{bibliografia}

\end{document}