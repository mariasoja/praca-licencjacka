\documentclass[12pt,a4paper]{report}
\linespread{1.5}
\usepackage[left=3cm,right=2.5cm,top=2.5cm,bottom=2.5cm]{geometry}
\usepackage[utf8]{inputenc}
\usepackage{polski}
\usepackage{amsmath}
\usepackage{centernot}
\usepackage{amsfonts}
\usepackage{amssymb}
\usepackage{amsthm}
\usepackage{enumerate}
\usepackage{natbib} %bibtex

\newtheorem{definition}{Definicja}[chapter]
\newtheorem{przyklad}{Przykład}

\newtheorem{tw}[definition]{Twierdzenie}
\newtheorem{remark}[definition]{Uwaga}

\author{Maria Soja}
\title{Teoria Stevensa pomiaru statystycznego}

\newcommand{\parauporzadkowana}[2]{\left\langle {#1}; {#2} \right\rangle}
\newcommand{\zbior}[1]{\left\lbrace {#1} \right\rbrace }
\newcommand{\domkniecie}[1]{\left[ {#1} \right] }
\newcommand{\notiff}{%
  \mathrel{{\ooalign{\hidewidth$\not\phantom{"}$\hidewidth\cr$\iff$}}}}
\newcommand{\tuple}[1]{\left\langle {#1} \right\rangle}
\begin{document}

\begin{titlepage}
\begin{flushleft}
\end{flushleft}
\begin{center}
\textsc{{\huge Politechnika Łódzka}}
\end{center}
\bigskip
\bigskip
\begin{center}
\textsc{{\Large Wydział Fizyki Technicznej, Informatyki i~Matematyki Stosowanej}}
\end{center}
\bigskip
\bigskip
\begin{Large}
Kierunek: Matematyka 
\\Specjalność: Matematyczne metody analizy danych biznesowych

\end{Large}
\bigskip
\bigskip
\noindent\hrulefill
\begin{center}
{\textbf{{\Large Teoria Stevensa pomiaru statystycznego.}}}
\end{center}
\begin{flushright}
{\large 
Maria Soja

Nr albumu: 
210088
}
\end{flushright}
\noindent\hrulefill
\bigskip
\bigskip
\begin{center}
{\large Praca licencjacka
napisana w~Instytucie Matematyki 
\\Politechniki Łódzkiej 
\bigskip
\bigskip
\\Promotor: dr, mgr inż. Piotr Kowalski
 }
\end{center}
\bigskip
\bigskip
\bigskip
\bigskip
\begin{center}
{\textsc{\large Łódź, wrzesień 2019}}
\end{center}
\end{titlepage}


\tableofcontents

\chapter{Wstęp}

Tematem pracy jest teoria Stevensa pomiaru statystycznego. Teoria ta wiąże się nieodłącznie z pojęciem skali w zagadnieniach analizy danych. Pomimo tego, że to zagadnienie pojawia się w wielu książkach z zakresu analizy danych np. w {\citep{walesiak2009statystyczna}} to zaskakującym jest, że nigdzie nie jest ono matematycznie poprawnie wyjaśnione. Ponadto różne pobliskie narzędzia analizy danych są nieporównywalnie lepiej matematycznie opisane. W bibliografiach znajduje się w tych miejscach odniesienie do teorii Stevensa sformułowanej w pracy napisanej przez Adams, Fagot, Robinson {\citep{adams1965theory}}, która okazuje się być próbą wyjaśnienia nieprecyzyjności z wcześniejszych prac Stevensa. Publikacja pracy miała miejsce w roku 1965 co sugeruje, że odpowiedziała na większość z nękających społeczność pytań. Sama praca okazuje się zawierać znacznie więcej niż definicje skal pomiarowych. Autorzy starają się w niej sformułować bardzo ogólna teorie o formułowaniu wszelkich zdań z zakresu statystyki. W niniejszej pracy podjęta została próba omówienia choć części tej bogato przedstawionej tam teorii. 
 
Praca uporządkowana jest w następujący sposób. W rozdziale drugim znajdują się preliminaria, w których umieszczono miedzy innymi elementy algebraicznej teorii grup oraz teorii mnogości. W rozdziale trzecim omawiane są kolejne aspekty teorii Stevensa pomiaru statystycznego. Część ta rozpoczyna się od wprowadzenia numerical assignment systems (NAS), których zadaniem jest modelowanie rzeczywistych pomiarów wykonywanych w statystyce. W dalszej części wprowadzone są typy skal jako warunki nakładane na poszczególne NAS, w tym miejscu dokonujemy porównania ich założeń z teoria grup algebraicznych. Dalsza część pracy stanowi omówienie teorii niezmienniczości w pomiarach statystycznych. W pracy zaprezentowano kilka twierdzeń o warunkach równoważnych rożnych typów niezmienniczości. Pomimo tego, iż praca ta nie wyczerpuje i nie wyjaśnia w pełni teorii przedstawionej w pracy Adams, Fagot, Robinson {\citep{adams1965theory}}, wnosi istotną wartość poprzez zaprezentowanie własnych przykładów oraz dowodów twierdzeń zaprezentowanych w pracy, a które wyraźnie rzutują na rozumienie pojęć tej teorii. Ponadto pokazuje, iż tematyka przedstawiona ponad 50 lat temu, dalej jest nietrywialną i nie jest wprowadzana w pełni przez autorów książek z analizy danych z uwagi na jej ogólność i stopień skomplikowania. W rozdziale czwartym podsumujemy  matematyczne badania tej teorii i omówimy wypracowane wnioski.

\chapter{Preliminaria}


\section{Elementy teorii mnogości}
W naszej pracy rozważane będą zagadnienia wymagające doprecyzowania wielu pojęć z zakresu dziedzin matematyki, takich jak teoria mnogości czy algebra. 


Zdefiniujmy parę uporządkowaną. Warto powiedzieć, że w wielu pracach definicje tam użyte są matematycznie niepoprawne.

\begin{definition}[{\citep[Sec 3.3]{kuratowski1966wstkep}}]
Parą uporządkowana składającą się z poprzednika $a$ oraz następnika $b$ nazwiemy zbiór składający się z elementów poniżej opisanych i oznaczany:
\begin{equation*}
\parauporzadkowana{a}{b}=\zbior{\zbior{a}, \zbior{a,b}}.
\end{equation*}

\end{definition}

\begin{definition}[Iloczyn kartezjański{\citep[Sec 3.4]{kuratowski1966wstkep}}]
Iloczynem kartezjańskim zbiorów $X$ i $Y$ nazywamy zbiór wszystkich par uporządkowanych $\parauporzadkowana{x}{y}$, gdzie $x \in X$ i $y \in Y$. Zbiór ten oznaczamy przez $ X \times Y$; zatem
\begin{equation*}
X \times Y= \zbior{\parauporzadkowana{x}{y}:  x\in X  , y \in Y}.
\end{equation*} 

\end{definition}

\begin{definition}[{\citep[Sec 6.1 Def. 6.1]{kraszewski2007wstkep}}]
Dane są dwa zbiory $X$ i $Y$. Relacją $\mathcal{R}$ (dwuargumentową) między elementami zbioru $X$, a elementami zbioru $Y$ nazywamy dowolny podzbiór iloczynu kartezjańskiego $X \times Y$.
\end{definition}
Chcąc zapisać, że pewien element $x$ jest w relacji $\mathcal{R}$ z pewnym elementem $y$ piszemy
$$
x\mathcal{R}y.
$$

Najpowszechniej stosowaną relacją jest funkcja.

\begin{definition}[Funkcja{\citep[Sec 4.1]{kuratowski1966wstkep}}]
Niech dane bedą dwa niepuste zbiory $X$ i $Y$. Przez funkcję, której argumenty przebiegają zbiór $X$, wartości zaś należą do zbioru $Y$, rozumiemy każdy podzbiór $f$ iloczynu kartezjańskiego $X \times Y$ o tej własności, że dla każdego $x \in X$ istnieje jeden i tylko jeden $y$ taki, że $\parauporzadkowana{x}{y} \in f$. 

\end{definition}

Równoważnie zapisujemy, że $f(x)=y$

\begin{definition}[Superpozycja funkcji{\citep[Sec 4.2 Def. 1.]{kuratowski1966wstkep}}]
Niech dane będą trzy zbiory $X$, $Y$ i $Z$ oraz dwie funkcje $f:X\to Y$, oraz $g:Y\to Z$. Funkcje te wyznaczają trzecia funkcję złożoną $h:X\to Z$ (nazywaną superpozycją funkcji $f$ i $g$) określoną przez warunek
\begin{equation*}
\forall_{x \in X} \quad h(x)=g(f(x)).
\end{equation*}
\end{definition}

\begin{definition}[{\citep[Sec 5.2 Def. 5.5]{kraszewski2007wstkep}}]
Niech $f:X \to Y$.
\begin{enumerate}
\item
Mówimy, że funkcja $f$ jest różnowartościowa i piszemy $f:X\xrightarrow{1-1} Y$, jeśli różnym argumentom przyporządkowuje ona różne wartości, czyli
\begin{equation*}
\forall{x_{1},x_{2}}\in X \quad x_{1}\ne x_{2} \implies f(x_{1})\ne f(x_{2}).
\end{equation*}
Funkcję taką nazywamy też injekcją lub mówimy, że jest $1-1$. Potocznie mówi się, że funkcja różnowartościowa nie skleja argumentów.
\item
Mówimy, że funkcja $f$ jest na i piszemy $f:X\xrightarrow{na}Y$, jeśli każdy element jej przeciwdziedziny jest wartością funkcji dla pewnego jej argumentu, czyli
\begin{equation*}
\forall_{y\in Y} \exists_{x\in X} \quad y=f(x).
\end{equation*}
Funkcję taką nazywamy też surjekcją.
\item
Jeśli funkcja $f$ jest różnowartościowa i na, nazywamy ją wzajemnie jednoznaczną i piszemy $f:X\xrightarrow[na]{1-1} Y$. O takiej funkcji mówimy też, że jest bijekcją.
\end{enumerate}
\end{definition}


\section{Teoria grup algebraicznych}

Bazowym pojęciem algebry jest pojęcie działania.

\begin{definition}[Działanie{\citep[Sec 4.1]{jedrzejewski2011algebra}}]
Niech $X$ będzie dowolnym niepustym zbiorem. Powiemy, że $\circ$ jest działaniem w zbiorze $X$ jeśli $\circ: X \times X \to X$.
\end{definition}


\begin{definition}[{\citep[Sec 4.1 Def. 4.3]{jedrzejewski2011algebra}}]
Działanie $\circ$ w zbiorze $X$ nazywamy działaniem łącznym, jeśli spełniony jest warunek 
\begin{equation*}
(a\circ b)\circ c = a\circ(b\circ c)
\end{equation*} 
dla każdych elementów $a,b,c$ ze zbioru $X$.
\end{definition}
\begin{definition}[{\citep[Sec 4.1 Def. 4.4]{jedrzejewski2011algebra}}]
Działanie $\circ$ w zbiorze $X$ nazywamy działaniem przemiennym, jeśli spełnia ono warunek
\begin{equation*}
a\circ b=b\circ a
\end{equation*}
dla każdych elementów $a,b$ ze zbioru $X$.
\end{definition}
Przykładem działania jest dodawanie liczb naturalnych w zbiorze $N$.
\begin{definition}[{\citep[Sec 4.1 Def. 4.8]{jedrzejewski2011algebra}}]
Element $e$ nazywamy elementem neutralnym w zbiorze $X$, jeśli 
\begin{equation*}
e\circ x = x\circ e = x
\end{equation*}
dla każdego elementu $x$ ze zbioru $X$.
\end{definition}


\begin{definition}[{\citep[Sec 4.2]{jedrzejewski2011algebra}}]
W teorii tej występują następujące struktury algebraiczne.
\begin{itemize}
\item
Strukturę algebraiczną, złożoną z niepustego zbioru $X$ i jednego działania łącznego, nazywamy półgrupą.
\item
Półgrupę, w której istnieje element neutralny, nazywamy półgrupą z jedynką lub monoidem. Półgrupę ,w której działanie jest przemienne, nazywamy półgrupę przemienną lub półgrupę abelową. 
\item
Półgrupę z jedynką gdy dla dow. $a \in X$ istnieje element odwrotny $a'$ tzn. taki, że 
$$
a'\circ a = a \circ a' =e
$$
nazywamy grupą.
\end{itemize}
\end{definition}




\chapter{Teoria Pomiaru Statystycznego Stevensa}

W poniższym rozdziale będziemy korzystać z artykułu {\citep{adams1965theory}}.

Standardowo w analizie pomiaru, wynik będziemy przedstawiać jako funkcję o wartości liczbowej. Pomiary masy w funtach będą reprezentowane przez funkcję, która może być oznaczona jako "$Ib$";  ta funkcja jest skorelowana z każdym obiektem $x$ który można zważyć wartością liczbową, $Ib(x)$ - waga $x$ w funtach. Dowolną funkcję tego rodzaju nazywamy numerical assignment. Choć w języku polskim nie jest dostępne utarte tłumaczenie tego pojęcia rozumiemy je jako formę przypisania liczby do danego obiektu. Na przykład przy pomiarze wagi możemy użyć rożnych jednostek. Jest to formalnie przedstawione  przez klasę przypisań liczbowych (funkcja funta, funkcja uncji, funkcja tona, itp.). Mówiąc ogólnie o pomiarze, przyjmiemy jako podstawowe pojęcie klasę przypisań liczbowych. Drugim podstawowym pojęciem jest permissible transformation czyli funkcja odwzorowującą wartości jednego numerical assignment danej klasy na wartości innego numerical assignment. Te dwa pojęcia składają się na numerical assignment system. Zdefiniujmy formalnie czym jest numerical assignment system w ujęciu pracy {\citep{adams1965theory}}.
\begin{definition}[Numerical assignment system (NAS){\citep[Def. 1]{adams1965theory}}] 
Numerical assignment system to uporządkowana czwórka $\tuple{A, M, a, \Phi}$ spełniająca poniższe warunki:
\begin{enumerate}
\item
$A$ jest niepustym zbiorem, $a$ jest niepustym podzbiorem liczb rzeczywistych, $M$ jest zbiorem funkcji przekształcających $A$ w $a$, i $\Phi$ jest klasą funkcji przekształcających $a$ w siebie.  
\item
Dla wszystkich $m$ z $M$ i $\phi$ z $\Phi$,
$$
 \phi \circ m \in M.
$$ 
\item
$\Phi$ zawiera tożsamościową transformację (identyczność) oraz zachodzi
$$
\forall_{\phi_{1}, \phi_{2} \in \Phi} \quad \phi_{1} \circ \phi_{2} \in \Phi.
$$

\end{enumerate}

\end{definition}

\begin{remark}[O warunku 2]
Warunek 2. powyższej definicji jest po prostu wymogiem na to aby każde permissible transormation przenosiło dowolne numerical assignment do innego numerical assignment tego samego systemu.
\end{remark}

\begin{remark}[O warunku 3]
Warunek 3. nie jest kluczowy, ale może być wymuszany, ponieważ tożsamościowe transformacja zdecydowanie przypisuje numerical assignment do numerical assignment. Poza tym musimy pamiętać, że złożenie dwóch transformacji musi zawsze dawać permissible transformation.
\end{remark}

  Można zauważyć, że pewne rzeczy przyjmowane za pewnik dotyczące numerical assignment i permissible transfomations  nie są zakładane w powyższej definicji. Po pierwsze, w teorii tej nie zakładamy, że permissible transfomations stanowią grupę, w rozumowaniu algebry przypomnianej w definicji 2.13 wymaga ona  dodatkowych założeń, np. że dla każdego $\phi$ z $\Phi$, że $\Phi$ zawiera odwrotność $\phi$, a to z kolei wymagałoby żeby dodatkowo wszystkie $\phi$ z $\Phi$ były odwzorowaniem różnowartościowym $a$ w siebie. Autorzy pracy {\citep{adams1965theory}} nie zakładali, że numerical assignment z $M$ odwzorowuje $A$ na $a$. To znaczy, że pomiary nie przyjmują wszystkie możliwe wartości liczbowe. Wreszcie co może wydawać najpoważniejszym pominięciem, nie zakładamy odwrotności warunku 2. w definicji, to znaczy, że każde numerical assignment z danego NAS może być przeniesione na inny za pomocą pewnej permissible transformation. W poniższych przykładach zilustrujemy dlaczego nie są uzasadnione mocniejsze założenia.

\begin{enumerate}
\item
\textbf{Wagi z różnym maksymalnym obciążeniem}. Jeżeli będziemy ważyć przedmiot którego waga jest większa od maksymalnego obciążenia jednej z wag wtedy wyniki naszych pomiarów będą się różnić. 
\item
\textbf{Termometry z różnymi maksymalnymi i minimalnymi stopniami}. Jeżeli temperatura będzie wyższa lub niższa niż skala jednego termometru wtedy wyniki naszych pomiarów również będą się różnić.
\end{enumerate}  
W teorii tej założone są zatem warunki dużo słabsze od przypomnianych grup algebraicznych. Tak słabe założenia mają uzasadnienie w zadaniach statystycznych. Stanowi jednak duży problem w opracowaniu matematycznych następstw. Z uwagi na takie problemy rozważany jest często NAS regularny. Jego założenia są wypośrodkowane pomiędzy zwykłym NASem a tym co zakładamy o grupach algebraicznych.  
\begin{definition}[{\citep[Def. 2]{adams1965theory}}]
Numerical assignment system $U=\tuple{A, M, a, \Phi}$ jest regularny wtedy i tylko wtedy gdy spełnione są poniższe warunki:
\begin{enumerate}
\item
Elementy z $M$ i $\Phi$ są różnowartościowe.
\item
Istnieje $m_{0}\in M$, $m_0:A\to a$ i które generuje $M$ w tym sensie, że dla każdego $m\in M$ istnieje takie $\phi\in \Phi$, że
$$
\phi \circ m_{0}=m.
$$
\item
Dla wszystkich skończonych $n$ i elementów $\alpha_{1},\dots,\alpha_{n}\in a$ i  $\phi\in \Phi$, istnieje taka funkcja $\Psi\in \Phi$, $\Psi:a\to a$, że dla każdego $i=1,\ldots,n$ 
$$
\Psi(\Phi(\alpha_{i}))=\alpha_{i}.
$$
Warunek ten nazywa się skończoną odwrotnością.
\end{enumerate}
\end{definition}
\section{Typy skal}
Definicje Stevensa typów skal są jednym z najważniejszych pojęć w jego teorii. Skala w teorii tej służy uporządkowaniu, określeniu różnych systemów NAS
\begin{definition}[{\citep[Def. 3]{adams1965theory}}]
Niech $U=\tuple{A, M, a, \Phi}$ będzie NAS. Wtedy powyższe $U$ jest:
\begin{enumerate}
\item
Skalą nominalną gddy $\Phi$ jest zbiorem wszystkich różnowartościowych funkcji odwzorowujących $a$ w siebie.
\item
Skalą porządkową gddy $\Phi$ jest zbiorem wszystkich ściśle rosnących odwzorowań $a$ w siebie.
\item
Skalą interwałową (nazywaną też przedziałową) gddy $a$ jest zbiorem wszystkich liczb rzeczywistych i $\Phi$ jest zbiorem wszystkich funkcji $\phi$, takich, że istnieją $\beta,\gamma$ gdzie $\beta>0$ i
\begin{equation*}
\phi(\alpha)=\beta\alpha+\gamma
\end{equation*}
dla wszystkich $\alpha$ z $a$.
\item
Skalą ilorazową (nazywaną też stosunkową) gddy $a$ jest zbiorem liczb dodatnich rzeczywistych i $\Phi$ jest zbiorem wszystkich przekształceń takich, że istnieje $\beta>0$ i
\begin{equation*}
\phi(\alpha)=\beta\alpha
\end{equation*}
dla wszystkich $\alpha$ z $a$.
\end{enumerate}
\end{definition}

Zauważmy, że skale w powyższej definicji są uporządkowane od najsłabszej do najmocniejszej czyli dla dowolnej skali spełnione są założenia skal słabszych.
\begin{przyklad}
Przykładem zmiennej w skali nominalnej jest płeć gdzie mamy dwie wartości, kobieta i mężczyzna którym przykładowo można przypisać odpowiednio 1 i 0. W skali porządkowej zmienną może być stopień wysmażenia mięsa czyli przykładowo słabo, średnio i mocno, zmienne te możemy uporządkować i przypisać odpowiednio wartości 1, 2 i 3. W skali interwałowej różne pomiary temperatury możemy uporządkować oraz obliczyć jaka jest różnica pomiędzy pomiarami i przypisać im odpowiednie wartości liczbowe.
\end{przyklad}

%\begin{przyklad}
%Rozważmy zbiór nazw dni tygodnia czyli
 % $$
%A=\zbior{\textrm{Poniedziałek, Wtorek, Środa, Czwartek, Piątek, Sobota, Niedziela}}
%$$
%Dniom tygodnia przypiszemy liczby $a=\zbior{0,1,2,3,4,5,6,7,8,9,10}$
%\end{przyklad}

%Bardzo trudno jest jawnie podać postać rodziny $M$ najczęściej odbywa się to w ten sposób, że zakłada się pewne konkretne przekształcenie należy do tej rodziny $M$ a inne przekształcenia uzyskuje się przez jego superpozycje z różnymi funkcjami z $\Phi$ 

%\begin{definition}[Przekształcenia generujące rodzinę $M$]
%Niech $k \in N$ rozważmy 
%$$
%M_{0} =\zbior {m_{1}, m_{2}, \dots ,m_{k}, \forall_{i \in {1, \dots ,k}}\quad m_{i}: A \to a}
%$$ 
%zdefiniujmy rekurencyjną formułę 
%$$
%M_{k+1} = M_{k} \cup \zbior{m; m:A \to a , \exists_{\overline{m} \in M_{k}} \exists_{\phi \in \Phi} \quad m= \phi \circ \overline{m}}
%$$ 
%wtedy $M= \bigcup^{\infty}_{n=0}M_{n}$ nazywamy rodziną przypisań %wygenerowana przez $m_{1},\dots, m_{k}$ oraz rodzinę $\Phi$.
%\end{definition}

%\begin{tw}
%Dla zbioru $A$ z Przykładu 1, rozważmy
 %\begin{equation*}
%\begin{split}
%m_{0} & =\zbior{\parauporzadkowana{\textrm{Poniedziałek}}{0},\parauporzadkowana{\textrm{Wtorek}}{1},\parauporzadkowana{\textrm{Środa}}{2},\parauporzadkowana{\textrm{Czwartek}}{3},\parauporzadkowana{\textrm{Piątek}}{4},\\
%& \parauporzadkowana{\textrm{Sobota}}{5},\parauporzadkowana{\textrm{Niedziela}}{6}} \left. \rigth\rbrace
%\end{split}
%\end{equation*}

%i niech $M$ będzie zbiorem generowanym przez $m_{0}$ oraz rodzinę $\Phi \colon A \to a$ oraz $\Phi$ jest klasą funkcji różnowartościowych przekształcających $a$ w siebie wtedy taki układ $U$ jest skalą nominalną oraz zachodzi poniższa równoważność
 %$$
%\forall_{y,z \in A} y=z \iff m(y)=m(z).
%$$
%\end{tw}
%\begin{proof}
%Rozważmy $y,z \in A$ i niech $m \in M$. Wtedy
%\begin{equation*}
%\exists_{\phi \in \Phi} m=\phi \circ \overline{m}.
%\end{equation*}
%Jeśli $y=z$ to $\overline{m}(y)=\overline{m}(z)$. $\phi$ jest funkcją zatem 
%\begin{equation*}
%\phi \circ \overline{m}(y)=\phi \circ \overline{m}(z) \iff m(y)=m(z).
%\end{equation*}
%Jeśli $y\ne z$ to $\overline{m}(y)\ne \overline{m}(z)$. $\phi$ jest różnowartościowa zatem
%\begin{equation*}
%\phi \circ \overline{m}(y) \ne \phi \circ \overline{m}(z) \iff m(y) \ne m(z)
%\end{equation*}
%Przypuśćmy, że $m(y)=m(z) \land y \ne z$. Wobec powyższego sprzeczność. Zatem
%\begin{equation*}
%\forall_{y,z \in A} y=z \iff m(y)=m(z).
%\end{equation*}
%\end{proof}

%\begin{tw}
%Dla zbioru $A$ z Przykładu 1, rozważmy
 %\begin{equation*}
%\begin{split}
 %m_{0} & =\zbior{\parauporzadkowana{\textrm{Poniedziałek}}{0},\parauporzadkowana{\textrm{Wtorek}}{1},\parauporzadkowana{\textrm{Środa}}{2},\parauporzadkowana{\textrm{Czwartek}}{3},\parauporzadkowana{\textrm{Piątek}}{4}, \\
%&  \parauporzadkowana{\textrm{Sobota}}{5},\parauporzadkowana{\textrm{Niedziela}}{6}} 
%\end{split}
%\end{equation*}
 
% i niech  $M$ będzie zbiorem generowanym przez $m_{0}$ oraz rodzinę $\Phi \colon A \to a$ oraz $\Phi$ jest klasą funkcji różnowartościowych przekształcających $a$ w siebie wtedy taki układ $U$ jest skalą nominalną, ale porządki zdefiniowane przez przypisania z rodziny $M$ nie tworzą porządku w zbiorze $A$.
%\end{tw}
%\begin{proof}
%Przypuśćmy, że istnieje $\le$ będące porządkiem w $A$ zgodnym ze wszystkimi przypisaniami z $M$. Wtedy
%\begin{equation*}
%\forall_{m\in M} \quad y \le z \implies m(y) \le m(z).
%\end{equation*}
%Pokażemy sprzeczność. Niech $y\le z \implies \overline{m}(y)\le \overline{m}(z)$ oraz $y\ne z$. Rozważmy $\phi :a \to a$ takie, że 
%$$
%\forall_{v \in a} \phi(v) = 10 -v.
%$$ 
%Zauważmy, że $\phi$ jest funkcją liniową która nie jest stała, zatem jest różnowartościowa. Wtedy $y \le z \implies \overline{m}(y) \le \overline{m}(z)$. Skoro $\forall_{m \in M} y\le z \implies m(y) \le m(z)$ to 
%\begin{equation*}
%\begin{split}
%\phi \circ \overline{m}(y) \le \phi \circ \overline{m}(z) & \iff 10 -\overline{m}(y) \le 10 - \overline{m}(z)\iff -\overline{m}(y) \le -\overline{m}(z) \\
%\end{split}
%\end{equation*}
%\begin{equation*}
%\overline{m}(y) \le \overline{m}(z) \land \overline{m}(y) \ge \overline{m}(z)
%\end{equation*}
%zatem $y=z$ co jest sprzeczne z przypuszczeniem.

%\end{proof}

%\begin{przyklad}
%Rozważmy zbiór 
%$$
%A=\zbior{\textrm{bardzo niski, niski, średni, wysoki, bardzo wysoki}}
%$$
 %oraz przypiszmy mu zbiór liczb  $a=\zbior{1,2,3,4,5,6,7,8,9,10}$. 
%\end{przyklad}
%\begin{tw}
%Dla zbioru $A$ z przykładu 2, rozważmy 
%$$
%m_{0}=\zbior{\parauporzadkowana{\textrm{bardzo niski}}{1},\parauporzadkowana{\textrm{niski}}{2},\parauporzadkowana{\textrm{średni}}{3},\parauporzadkowana{\textrm{wysoki}}{4},\parauporzadkowana{\textrm{bardzo wysoki}}{5}}
%$$
%i niech $M$ będzie zbiorem generowanym przez $m_{0}$ oraz rodzinę $\Phi \colon A \to a$ oraz $\Phi$ jest zbiorem ściśle monotonicznych rosnących odwzorowań $a$ w siebie wtedy taki układ $U$ jest skalą porządkową.
%\end{tw}

%\begin{przyklad}
%Rozważmy zbiór gdzie zmienna jest rokiem urodzenia czyli niech $A=\zbior{1980,1995,1997,2000,2009}$ oraz przypiszmy mu liczby $a=\zbior{1,2,3,4,5}$.
%\end{przyklad}
%\begin{tw}
%Dla zbioru z przykładu 3, rozważmy 
%$$
%m_{0}=\zbior{\parauporzadkowana{1980}{1},\parauporzadkowana{1995}{2},\parauporzadkowana{1997}{3},\parauporzadkowana{2000}{4},\parauporzadkowana{2009}{5}}
%$$
%gdzie $M$ jest zbiorem generowanym przez $m_{0}$ oraz rodzinę $\Phi \colon A \to a$ oraz $a$ jest zbiorem wszystkich liczb rzeczywistych, $\Phi$ jest zbiorem wszystkich funkcji $\phi$, takich że dla dow. $\beta,\gamma$ gdzie $\beta>0$
%\begin{equation*}
%\phi(\alpha)=\beta\alpha+\gamma
%\end{equation*}
%dla wszystkich $\alpha$ z $a$. Wtedy taki układ $U$ jest skalą interwałową w której nie można wykonać operacji mnożenia czyli nie można określić ile razy ktoś jest starszy.
%\end{tw}
%\begin{proof}
%Niech $y \in A$, niech $x$ oznacza wiek a $z$ obecny rok.
%\begin{equation*}
%x=z-y
%\end{equation*}
%Przypuśćmy, że 
%\begin{equation*}
%\forall_{m \in M} \forall_{n \in N}\quad ny=nx \implies nm(y)=nm(x)
%\end{equation*}
%Pokażemy sprzeczność. Zauważmy, że $ny=n(z-y)$ $\implies$ $nm(y)=n(m(z)-m(y))$ oraz $z \ne y$, $z>y$ zatem
%\begin{equation*}
%nm(y)=nm(z)-nm(y) \iff 2nm(y)=nm(z) \iff 2m(y)=m(z)
%\end{equation*}
%Weźmy $y=2000$ oraz $z=2019$ wstawiając do powyższego mamy %4000=2019. Sprzeczność zatem
%\begin{equation*}
%\forall_{m \in M} \forall_{n \in N}\quad ny\ne nx \implies nm(y)\ne nm(x)
%\end{equation*}
%\end{proof}
 
%\begin{przyklad}
%Rozważmy zbiór, gdzie zmienną jest wysokość mierzona w centymetrach czyli niech $A=\zbior{150,155,160,165,170,175}$ oraz przypiszmy liczby $a=\zbior{0,1,2,3,4,5}$. 
%\end{przyklad}
%\begin{tw}
%Dla zbioru $A$ z przykładu 4, rozważmy
%$$
%m_{0}=\zbior{\parauporzadkowana{150}{0},\parauporzadkowana{155}{1},\parauporzadkowana{160}{2},\parauporzadkowana{165}{3},\parauporzadkowana{170}{4},\parauporzadkowana{175}{5}}
%$$
 %gdzie $M$ jest zbiorem generowanym przez $m_{0}$ oraz rodzinę $\Phi \colon A \to a$ oraz $a$ jest zbiorem wszystkich liczb dodatnich rzeczywistych oraz $\Phi$ jest zbiorem wszystkich przekształceń takich że dla dow. $\beta>0$
%\begin{equation*}
%\phi(\alpha)=\beta\alpha
%\end{equation*}
%dla wszystkich $\alpha$ z $a$. Wtedy taki układ $U$ jest skalą ilorazową.
%\end{tw}
%W skali ilorazowej możliwe jest dokonywanie wszystkich operacji matematycznych.






\section{Operacje statystyczne}
 Sformułujemy jeszcze jedną wstępną koncepcje tj. operację statystyczną lub bardziej ogólnie, operację matematyczną na pomiarach. W obliczeniach statystycznych wykonuje się bardzo wiele różnych operacji. Należałoby zatem sformułować bardzo ogólną definicje działań matematycznych, która by uwzględniała wszystkie poszczególne przypadki. Skupmy się zatem na specjalnej klasie działań matematycznych i statystycznych, w których wynik jest obliczany ze skończonej liczby pomiarów (liczba ta nie musi być ustalona z góry) w pewien jednakowy sposób. Zawiera to w szczególnych przypadkach dobrze znane operacje jak np. obliczenie średniej, mediany, odchylenia standardowego, a także bardziej elementarne działania matematyczne takie jak na przykład dodawanie i odejmowanie. Nie zawiera natomiast działań stosowanych do nieskończonych sekwencji. Wydaje się że, standardowe operacje statystyczne mogą być przedstawione jako specjalne przypadki uogólnionej funkcji rzeczywistej. Posłużmy się następującym pojęciem pomocniczym. 
Poniższe dwie definicje zostały opracowane na podstawie {\cite[Sec. 4]{adams1965theory}}
\begin{definition}[Domknięcie zbioru skończonymi ciągami]
Niech $A$ będzie dowolnym niepustym zbiorem. Oznaczmy przez $S(A)$ zbiór wszystkich skończonych ciągów o elementach w $A$, tzn.
$$
S(A)=\zbior{(a_n)^{N}_{n=1}; N\in \mathbb{N}, \forall_{n\in \zbior{1,\dots,N}} a_n \in A}.
$$
Oznaczmy dalej przez $A_1 =A$, natomiast przez $A_2 = S(A_1) \cup A_1$. Ogólnie niech $A_n = S(A_{n-1})\cup A_{n-1}, n>1$. Wtedy domknięciem zbioru $A$ skończonymi ciągami nazywamy
$$
\domkniecie{A}=\bigcup^{\infty}_{n=1} A_n.
$$ 
\end{definition}
Domknięcie w pewien szczególny sposób przenosi się również na funkcje.
\begin{definition}[Rozszerzenie funkcji na skończone ciągi]
Niech $A$ będzie dowolnym niepustym zbiorem. Niech $m:A\to \mathbb{R}$. Wtedy funkcja $\domkniecie{m}:\domkniecie{A}\to \domkniecie{\mathbb{R}}$ dana jest formułą, że
\begin{itemize}
\item jeśli $x\in A$ to $\domkniecie{m}(x) =m(x) \in \mathbb{R}$
\item jeśli $x\notin A$ to jest ciągiem elementów z $\domkniecie{A}$. Oznaczmy $x=(x_1,\dots,x_N)$. Wtedy 
$$
\domkniecie{m}(x)=\domkniecie{m}((x_1,x_2,\dots,x_N))=(\domkniecie{m}(x_1),\domkniecie{m}(x_2),\dots, \domkniecie{m}(x_N))\in \domkniecie{\mathbb{R}}.
$$
\end{itemize}
Funkcja $\domkniecie{m}$ nazywana jest rozszerzeniem funkcji $m$ na skończone ciągi.
\end{definition}
Więc jeżeli $x$ jest dowolnym ciągiem, $\domkniecie{m}(x)$ jest po prostu odpowiadającym ciągiem wartości. Używając tego pojęcia możemy zdefiniować uogólnioną funkcję rzeczywistą której dziedzina jest podzbiorem $\domkniecie{\mathbb{R}}$. Większość konkretnych funkcji statystycznych, które chcemy uwzględnić mają zdecydowanie bardziej ograniczone dziedziny, ale pożądane jest rozważenie wszystkich funkcji tej klasy razem. Przykładem takich funkcji uogólnionych jest funkcja, której dziedziną są wszystkie uporządkowane pary liczb rzeczywistych wykonująca działanie dodawania par lub działanie, które znajduje różnicę dwóch średnich. Może być rozumiane jako działanie posiadające dziedzinę składającą się klasy wszystkich uporządkowanych par skończonego ciągu liczb rzeczywistych. Uogólnione działanie na pomiarach jest intuicyjnie wynikiem zastosowania uogólnionej funkcji rzeczywistej  do liczb przypisanych przez numerical assignment. Zatem uogólnione operacje na pomiarach są generowane przez odpowiednie uogólnione funkcje rzeczywiste. Formalne definicje tych dwóch pojęć są następujące.
\begin{definition}[{\citep[Def. 4.1]{adams1965theory}}]
Niech $\mathcal{D}\subset \domkniecie{\mathbb{R}}$. Wtedy funkcję $\mathcal{F}:\mathcal{D}\to \domkniecie{\mathbb{R}}$ nazywamy uogólnioną funkcją rzeczywistą.
\end{definition}
\begin{przyklad}
Przykładem takiej funkcji jest wzięcie średniej  liczb dowolnego ciągu czyli niech $\mathcal{D}=S(\mathbb{R})$, $Mean: S(\mathbb{R})\to \domkniecie{\mathbb{R}}\supset \mathbb{R}$
$$
\forall_{n \in \mathbb{N}} \forall_{(\alpha_1,\dots,\alpha_n)\in S(\mathbb{R})} \quad Mean((\alpha_1,\dots,\alpha_n))=\frac{\alpha_1+\dots+\alpha_n}{n}.
$$
\end{przyklad}
\begin{przyklad}
Kolejnym przykładem takiej funkcji jest funkcja sortująca liczby.
$$
Sort((5,1,7,8,0))=(0,1,5,7,8).
$$
Albo funkcja która filtruje liczby ujemne.
$$
FN((-1,2,7,-3,-4,5,6))=(2,7,5,6)
$$
Dobrym przykładem jest też funkcja która dodaje do każdej wartości liczbę 2.
$$
ADD2((1,3,5,0))=(3,5,7,2)
$$
\end{przyklad}
\begin{definition}[{\citep[Def. 4.2]{adams1965theory}}]
Niech $\mathcal{F}:\mathcal{D}\to \domkniecie{\mathbb{R}}$, $\mathcal{D}\subset\domkniecie{\mathbb{R}}$ Wtedy
$$
D=\zbior{\parauporzadkowana{m}{x}; m:A\to \mathbb{R}, x\in \domkniecie{A}, \domkniecie{m}(x)\in \mathcal{D}}.
$$
Funkcję $F:D\to\domkniecie{\mathbb{R}}$ opisaną formułą
$$
F:\parauporzadkowana{m}{x}\longmapsto\mathcal{F}(\domkniecie{m}(x)),
$$
nazywamy uogólnionym działaniem na pomiarach.
\end{definition}
Należy zauważyć, że w definicji uogólnionej funkcji rzeczywistej, nie określono dokładnej przeciwdziedziny dla takich funkcji. Funkcje te zatem mogą być bardzo dowolne. W dalszej części będziemy się zajmować głównie funkcjami, których przeciwdziedziny są liczby rzeczywiste.  Funkcje te służą do formułowania statystycznych wniosków. Teoria dalej skupia się wokół wprowadzenia koncepcji niezmienniczości tych funkcji. Tj. niezależność zdań statystycznych od przyjętej techniki pomiaru.

\section{Niezmienniczość statystycznych operacji}
Definiujemy trzy rodzaje niezmienniczości dla uogólnionych operacji na pomiarach względem NAS. 
\begin{definition}[{\citep[Def. 5]{adams1965theory}}]
Niech $U=\tuple{A,M,a,\Phi}$ będzie NAS $\mathcal{F}:\mathcal{D}\to\domkniecie{\mathbb{R}}$, $\mathcal{D}\subset \domkniecie{\mathbb{R}}$ i $F$ jest uogólnioną operacją na pomiarach odpowiadających $\mathcal{F}$. Wtedy:
\begin{enumerate}
\item
$F$ jest bezwzględnie niezmiennicza względem $U$ gddy
$$
\forall_{m_{1}, m_{2} \in M}  \forall_{\overline{x} \in \domkniecie{A}} \quad \domkniecie{m_1}(\overline{x}), \domkniecie{m_2}(\overline{x})\in \mathcal{D} \implies F(m_{1};\overline{x})=F(m_{2};\overline{x}).
$$
\item
$F$ jest odnośnikowo niezmiennicza względem $U$ gddy dla dow. $m_{1}$, $m_{2} \in M$ $x \in A$ i  $\overline{x} \in \domkniecie{A}$ o ile  $\domkniecie{m_1}(\overline{x})$, $\domkniecie{m_2}(\overline{x})\in \mathcal{D}$ to mamy.
$$
 m_{1}(x)=F(m_{1};\overline{x}) \iff m_{2}(x)=F(m_{2};\overline{x}).
$$
\item
$F$ jest porównawczo niezmiennicza względem $U$ gddy dla dow. $m_1$, $m_2 \in M$ i $\overline{x}$, $\overline{y} \in \domkniecie{A}$ o ile $\domkniecie{m_1}(\overline{x}), \domkniecie{ m_2}(\overline{x}), \domkniecie{m_1}(\overline{y}), \domkniecie{m_2}(\overline{y}) \in \mathcal{D}$ to mamy:
$$
F(m_{1};\overline{x})=F(m_{1};\overline{y}) \iff F(m_{2};\overline{x})=F(m_{2};\overline{y}).
$$
\end{enumerate} 
\end{definition}

Kolejne twierdzenia ograniczone są do regularnych NASów.

\begin{przyklad}
Bezwzględnie niezmiennicza względem $U$ jest np. funkcja:
$$
F(m;\overline{x})=Count(m;\overline{x}).
$$
Funkcja ta bierze dowolny ciąg i liczy ile jest w nim różnych elementów
\end{przyklad}
\begin{przyklad}
Odnośnikowo niezmiennicza w stosunku do skal interwałowych są np. funkcje
$$
F(m;\overline{x})=Mean(m;\overline{x}),
$$ 
które liczą średnią dowolnego ciągu.

Istotnie. Niech $U$ będzie skalą interwałową oraz NAS regularny.
Niech $m_1,m_2\in M$ i niech $x \in A $, $\overline{x}\in \domkniecie{A}$. Z regularności $U$ istnieje $m_0 \in M$, że
$$
m_1 = \phi_1 \circ m_0, \quad m_2=\phi_2\circ m_0
$$
Ponadto gdyż $U$ jest interwałowa to istnieją takie $\beta_{1}, \gamma_{1}, \beta_{2}, \gamma_{2}$, że
$$
\phi_1(\alpha)=\beta_1 \alpha + \gamma_1, \quad \phi_2(\alpha)=\beta_2 \alpha + \gamma_2.
$$
Niech $m_(x)=F(m;\overline{x})$. Wtedy
\begin{eqnarray*}
\phi_1\circ m_0(x) & = & Mean(\phi_1\circ m_0;\overline{x}). 
\end{eqnarray*}
Z postaci $\phi_{1}$ i własności średniej
\begin{eqnarray*}
\beta_1m_0(x)+ \gamma_1 & = & \beta_1 Mean(m_0;\overline{x}) + \gamma_1, \\
\beta_1m_0(x) & = & \beta_1 Mean(m_0;\overline{x}), \\
m_0(x) & = & Mean(m_0;\overline{x}), \\
\beta_2m_0(x) & = & Mean(\beta_2 m_0;\overline{x}), \\
\beta_2m_0(x) + \gamma_2 & = & Mean(\beta_2 m_0 + \gamma_2;\overline{x}),  \\
m_2(x) & = & F(m_2;\overline{x}) .
\end{eqnarray*}
Zatem
$$
m_1(x)=F(m_1;\overline{x}) \implies m_2(x)=F(m_2;\overline{x}).
$$
Wobec symetrii oznaczeń mamy
$$
m_2(x)=F(m_2;\overline{x}) \implies m_1(x)=F(m_1;\overline{x}).
$$
Skąd
$$
m_1(x)=F(m_1;\overline{x}) \iff m_2(x)=F(m_2;\overline{x}).
$$
Z dowolności $m_1,m_2$ oraz $x$, $F$ jest odnośnikowo niezmiennicza względem $U$.
\end{przyklad}
\begin{przyklad}
Porównawczo niezmiennicza w stosunku do skal porządkowych są funkcje maximum oraz minimum z próby.
\end{przyklad}


\begin{tw}[Warunki równoważne niezmienniczości bezwzględnej {\citep[Tw. 1]{adams1965theory}}]
Niech $U=\tuple{A,M,a,\Phi}$ będzie regularnym NAS, niech $\mathcal{F}:\mathcal{D}\to \domkniecie{\mathbb{R}}$, $\mathcal{D} \in \domkniecie{\mathbb{R}}$ i niech $F$ będzie uogólnioną operacją na pomiarach odpowiadającą $\mathcal{F}$. Wtedy:

\item
NWSR.
\begin{enumerate}
\item
$F$ jest niezmiennicza bezwzględnie w stosunku do $U$, 
\item
$$
\forall_{m \in M}\forall_{\phi \in \Phi} \forall_{\overline{x} \in \domkniecie{A}} \quad \domkniecie{m}(\overline{x}),\domkniecie{\phi \circ m}(\overline{x})) \in \mathcal{D} \implies \mathcal{F}( \domkniecie{\phi \circ m}(\overline{x}))=\mathcal{F}(\domkniecie{m}(\overline{x})).
$$
\item
$$
\forall_{\phi \in \Phi} \forall_{\overline{\alpha} \in \domkniecie{a}} \quad \overline{\alpha}, \domkniecie{\phi}(\overline{\alpha})\in \mathcal{D} \implies \mathcal{F}(\domkniecie{\phi}(\overline{\alpha}))=\mathcal{F}(\overline{\alpha}).
$$
\end{enumerate}

\end{tw}
\begin{proof}
Udowodnimy (1)$\implies$ (2)
Niech $F$ będzie bezwzględnie niezmiennicza w stosunku do $U$
\begin{equation}
\forall_{m_{1}, m_{2} \in M}  \forall_{\overline{x} \in \domkniecie{A}} \quad \domkniecie{m_1}(\overline{x}), \domkniecie{m_2}(\overline{x})\in \mathcal{D} \implies F(m_{1};\overline{x})=F(m_{2};\overline{x}).
\end{equation}
Niech $m \in M, \phi \in \Phi$ i $\overline{x} \in \domkniecie{A}$. Wtedy
$$
\mathcal{F}(\domkniecie{\phi \circ m}(\overline{x}))=F(\phi \circ m;\overline{x})
$$
Oczywiście $\phi \circ m \in M$. Zatem z (3.1)
$$
F(\phi \circ m; \overline{x})=F(m;\overline{x})=\mathcal{F}(\domkniecie{m}(\overline{x})).
$$

Udowodnimy (2)$\implies$ (1)
Niech $m_1,m_2 \in M$, niech $\overline{x} \in\domkniecie{A}$.
Z regularności $U$, istnieje $m_0\in M$ takie, że
$$
m_1=\phi_1\circ m_0, \quad m_2=\phi_2\circ m_0.
$$
Wtedy wyznaczają $\overline{\alpha}=m_0(\overline{x})$.
Ponadto z regularności $U$ istnieje $\Psi:a\to a$, że
$$
 \Psi(\phi_2(\overline{\alpha}))=\overline{\alpha}.
$$
Niech $m:=\phi_2\circ m_0$ oraz $\phi:=\phi_1\circ\Psi$, $\overline{x}=\overline{x}$. Wtedy
\begin{eqnarray*}
\mathcal{F}(\domkniecie{\phi_1\circ \Psi\circ \phi_2 \circ m_0}(\overline{x})) & = & \mathcal{F}(\domkniecie{m_2}(\overline{x})), \\
\mathcal{F}(\domkniecie{\phi_1\circ(\Psi\circ \phi_2)\circ m_0}(\overline{x})) & = & \mathcal{F}(\domkniecie{\phi_1\circ m_0}(\overline{x})) = \mathcal{F}(\domkniecie{m_1}(\overline{x})),\\
\mathcal{F}(\domkniecie{m_2}(\overline{x})) & = & \mathcal{F}(\domkniecie{m_1}(\overline{x})),\\
F(m_2;\overline{x}) & = & F(m_1;\overline{x}).
\end{eqnarray*}
Z dowolności $m_1,m_2$ oraz $\overline{x} \in \domkniecie{A}$ udowodniliśmy punkt (1).

Udowodnimy (3) $\implies$ (2)
Załóżmy, że
\begin{equation}
\forall_{\phi \in \Phi} \forall_{\overline{\alpha} \in \domkniecie{a}} \quad \overline{\alpha}, \domkniecie{\phi}(\overline{\alpha})\in \mathcal{D} \implies \mathcal{F}(\domkniecie{\phi}(\overline{\alpha}))=\mathcal{F}(\overline{\alpha}).
\end{equation}
Niech $m\in M, \phi \in \Phi$ i $\overline{x} \in\domkniecie{A}$, $\overline{\alpha}:=m(\overline{x})$. Z (3.2)
$$
 \mathcal{F}(\domkniecie{\phi\circ m}(\overline{x})))=\mathcal{F}(m(\overline{x})).
$$
Z dowolności wyboru $m$, $\phi$ oraz $\overline{x}$ udowodniliśmy punkt (2).

Udowodnimy $(2) \implies (3)$
Załóżmy, że
\begin{equation}
\forall_{m \in M}\forall_{\phi \in \Phi} \forall_{\overline{x} \in \domkniecie{A}} \quad \domkniecie{m}(\overline{x}),\domkniecie{\phi\circ m}(\overline{x}) \in \mathcal{D} \implies \mathcal{F}( \domkniecie{\phi \circ m}(\overline{x}))=\mathcal{F}(\domkniecie{m}(\overline{x})).
\end{equation}
Niech $\phi \in \Phi$, $\overline{\alpha} \in \domkniecie{a}$ oraz $\overline{\alpha}$, $\domkniecie{\phi}(\overline{\alpha})\in \mathcal{D}$. Niech $m \in M$, $\overline{x} \in \domkniecie{A}$, $\domkniecie{m}(\overline{x}):= \overline{\alpha}$, 
$$
 \mathcal{F}(\overline{\alpha})=\mathcal{F}(\domkniecie{m}(\overline{x})).
$$
Z (3.3)
$$
\mathcal{F}(\domkniecie{\phi \circ m}(\overline{x})=\mathcal{F}(\domkniecie{\phi}(\overline{\alpha})).
$$
Zatem mamy
$$
\mathcal{F}(\domkniecie{\phi}(\overline{\alpha}))=\mathcal{F}(\overline{\alpha}).
$$
Z dowolności wyboru $\phi$, $\overline{\alpha}$, $m$, $\overline{x}$ udowodniliśmy punkt (3).
\end{proof}

\begin{tw}[Warunki równoważne niezmienniczości odnośnikowej {\citep[Tw. 1]{adams1965theory}}]
Niech $U=\tuple{A,M,a,\Phi}$ będzie regularnym NAS, niech $\mathcal{F}:\mathcal{D}\to \domkniecie{\mathbb{R}}$ i niech $F$ będzie uogólnioną operacją na pomiarach odpowiadającą $\mathcal{F}$. Wtedy
NWSR
\begin{enumerate}
\item
$F$ jest odnośnikowo niezmiennicza w stosunku do $U$
\item
Dla dow. $m \in M$, $\phi \in \Phi$, $ x \in A$ i $\overline{x} \in \domkniecie{A}$ o ile $\domkniecie{m}(\overline{x}), \domkniecie{\phi\circ m}(\overline{x})) \in \mathcal{D}$ to mamy:
$$
(\domkniecie{\phi\circ m}(x))=\mathcal{F}( \domkniecie{\phi \circ m}(\overline{x}))) \iff m(x)=\mathcal{F}(\domkniecie{m}(\overline{x}))
$$
\item
$$
\forall_{\phi \in \Phi}\forall_{\alpha \in a}\forall_{\overline{\alpha} \in \domkniecie{a}} \quad \overline{\alpha}, \domkniecie{\phi}(\overline{\alpha}) \in \mathcal{D} \implies \big[\phi(\alpha)=\mathcal{F}(\domkniecie{\phi}(\overline{\alpha})) \iff \alpha=\mathcal{F}(\overline{\alpha})\big].
$$
\end{enumerate}

\end{tw}
\begin{proof}
Udowodnimy (1) $\implies$ (2)
Załóżmy, że dla dow. $m_{1}$, $ m_{2} \in M$, $x \in A$ i $\overline{x} \in \domkniecie{A}$ o ile $\domkniecie{m_1}(\overline{x}), \domkniecie{m_2}(\overline{x})\in \mathcal{D}$ to mamy:
\begin{equation}
 m_{1}(x)=F(m_{1};\overline{x}) \iff m_{2}(x)=F(m_{2};\overline{x}).
\end{equation}
Niech $m\in M, \phi \in \Phi$, $x \in A$ i $\overline{x}\in \domkniecie{A}$.
Stosujemy przekształcenia równoważne.
Dla $m_1:=\phi\circ m$,$ m_2:=m$, $x:=x$, $\overline{x}:=\overline{x}$ mamy
$$
\phi\circ m(x)=\mathcal{F}((\domkniecie{\phi \circ m}(\overline{x})) \iff \domkniecie{\phi \circ m}(x)=F(\phi \circ m;\overline{x}). 
$$
Z (3.4) mamy
$$
\iff m(x)=F(m;\overline{x}) \iff m(x)=\mathcal{F}(\domkniecie{m}(\overline{x})).
$$
Udowodnimy (2) $\implies (1)$
Załóżmy, że dla dow. $m \in M$, $\phi \in \Phi$, $ x \in A$ i $\overline{x} \in \domkniecie{A}$ o ile $\domkniecie{m}(\overline{x}), \domkniecie{\phi\circ m}(\overline{x})) \in \mathcal{D}$ to mamy:
\begin{equation}
(\domkniecie{\phi\circ m}(x))=\mathcal{F}( \domkniecie{\phi \circ m}(\overline{x}))) \iff m(x)=\mathcal{F}(\domkniecie{m}(\overline{x})).
\end{equation}
Niech $m_1,m_2 \in M$, $x \in A$, $\overline{x}\in \domkniecie{A}$ i niech $\domkniecie{m_1}(\overline{x})$, $\domkniecie{m_2}(\overline{x})\in \mathcal{D}$. Z regularności NAS istnieje $m_0$, $\phi_1$, $\phi_2$ takie, że
$$ 
m_1=\phi_1\circ m_0 \quad m_2=\phi_2\circ m_0.
$$
Niech $\alpha=m_0(x)$ i $\overline{\alpha}=\domkniecie{m_0}(\overline{x})$ wtedy ponownie z regularności NAS istnieje $\Psi$ takie, że dla każdego $\alpha_i \in \alpha$
$$
\Psi(\phi_2(\overline{\alpha_i}))=\overline{\alpha_i},
$$
oraz
$$
\Psi(\phi_2(\alpha))=\alpha.
$$
Z (3.5) dla $m:=\phi_2 \circ m_0$, $\phi:=\phi_1\circ \Psi$, $\overline{x}:=\overline{x}$, $x:=x$.
Równoważnie przekształcając
\begin{eqnarray*}
m_1(x) & = & F(m_1;\overline{x}), \\
 (\phi_1\circ) m_0(x) & = & F(\phi_1 \circ m_0;\overline{x}),\\
(\phi_1\circ\Psi\circ\phi_2\circ)m_0(x)&=&F(\phi_1\circ\Psi\circ\phi_2\circ m_0;\overline{x}). 
\end{eqnarray*}
Korzystając ponownie z (3.5)
$$
(\phi_2\circ )m_0(x)=F(\phi_2\circ m_0;\overline{x}), 
$$
$$
m_2(x)=F(m_2;\overline{x}).
$$
Z dowolności $m_1, m_2$, $x$ i $\overline{x}, F$ jest odnośnikowo niezmiennicza.

Udowodnimy (3) $\implies$ (2)
Załóżmy, że
\begin{equation}
\forall_{\phi \in \Phi}\forall_{\alpha \in a}\forall_{\overline{\alpha} \in \domkniecie{a}} \quad \overline{\alpha}, \domkniecie{\phi}(\overline{\alpha}) \in \mathcal{D} \implies \big[\phi(\alpha)=\mathcal{F}(\domkniecie{\phi}(\overline{\alpha})) \iff \alpha=\mathcal{F}(\overline{\alpha})\big].
\end{equation}
Niech $m\in M, \phi\in \Phi$, $x\in A$ i $\overline{x} \in \domkniecie{A}$ i niech $\overline{\alpha}$, $\domkniecie{\phi}(\overline{\alpha}) \in \mathcal{D}$. Dla  $\alpha:= m(x)$, $\overline{\alpha}:=\domkniecie{m}(\overline{x})$. Równoważnie
$$
\phi\circ m(x)=\mathcal{F}(\domkniecie{\phi\circ m}(\overline{x})), 
$$
$$
\phi(\alpha)=\mathcal{F}(\domkniecie{\phi}(\overline{\alpha})).
$$
Z (3.6)
$$
\alpha=\mathcal{F}(\overline{\alpha}), \quad m(x)=\mathcal{F}(\domkniecie{m}(\overline{x})).
$$
Z dowolności wyboru $m, \phi$ oraz $x$ udowodniliśmy punkt (2).

Udowodnimy $(2) \implies (3)$
Załóżmy, że dla dow. $m \in M$, $\phi \in \Phi$, $ x \in A$ i $\overline{x} \in \domkniecie{A}$ o ile $\domkniecie{m}(\overline{x}), \domkniecie{\phi\circ m}(\overline{x})) \in \mathcal{D}$ to mamy:
\begin{equation}
(\domkniecie{\phi\circ m}(x))=\mathcal{F}( \domkniecie{\phi \circ m}(\overline{x}))) \iff m(x)=\mathcal{F}(\domkniecie{m}(\overline{x}))
\end{equation}
Niech $\phi \in \Phi$, $\alpha \in a$, $\overline{\alpha} \in \domkniecie{a}$. Wtedy istnieje $x$ i $\overline{x}$ oraz takie $m$, że
$$
\alpha=m(x) \quad i \quad \overline{\alpha}=\domkniecie{m}(\overline{x}).
$$ 
Mamy
\begin{eqnarray*}
\phi(\alpha) & = & \mathcal{F}(\domkniecie{\phi}(\overline{\alpha})),\\
\phi \circ m(x)&=&\mathcal{F}(\domkniecie{\phi \circ m}(\overline{x})).
\end{eqnarray*}
Z (3.7) mamy
\begin{eqnarray*}
m(x) &=&\mathcal{F}(\domkniecie{m}(\overline{x})),\\
\alpha&=&\mathcal{F}(\overline{\alpha}).
\end{eqnarray*}
Z dowolności $\phi$, $\alpha$, $\overline{\alpha}$ udowodniliśmy punkt (3).
\end{proof}
\begin{tw}[Warunki równoważne niezmienniczości porównawczej {\citep[Tw. 1]{adams1965theory}}]
Niech $U=\tuple{A,M,a,\Phi}$ będzie regularnym NAS, niech $\mathcal{F}:\mathcal{D}\to \domkniecie{\mathbb{R}}$ i niech $F$ będzie uogólnioną operacją na pomiarach odpowiadającą $\mathcal{F}$. Wtedy
NWSR
\begin{enumerate}
\item
$F$ jest porównawczo niezmiennicza w stosunku do $U$.
\item
Dla dow. $m \in M$, $\phi \in \Phi$ i $\overline{x}$, $\overline{y} \in \domkniecie{A}$ o ile $\domkniecie{m}(\overline{x}), \domkniecie{
m}(\overline{y}), \domkniecie{\phi\circ m}(\overline{x})), \domkniecie{\phi\circ m}(\overline{y})) \in \mathcal{D}$ to mamy:

$$
\mathcal{F}(\domkniecie{\phi \circ m}(\overline{x}))=\mathcal{F}(\domkniecie{\phi \circ m}(\overline{y}))) \iff \mathcal{F}(\domkniecie{m}(\overline{x}))=\mathcal{F}(\domkniecie{m}(\overline{y})).
$$
\item
Dla dow. $\phi \in \Phi$ i $\overline{\alpha}$,$ \overline{\beta} \in \domkniecie{a}$ o ile $\overline{\alpha}, \overline{\beta}, \domkniecie{\phi}(\overline{\alpha}), \phi(\overline{\beta}) \in \mathcal{D}$ to mamy:
\begin{equation*}
\mathcal{F}(\domkniecie{\phi}(\overline{\alpha}))=\mathcal{F}(\domkniecie{\phi}(\overline{\beta})) \iff \mathcal{F}(\overline{\alpha})=\mathcal{F}(\overline{\beta}).
\end{equation*}
\end{enumerate}

\end{tw}
\begin{proof}
Udowodnimy $(1) \implies (2)$
Niech $F$ będzie porównawczo niezmiennicza oraz niech $m\in M, \phi \in \Phi$, $x, y \in A$ i $\overline{x}, \overline{y} \in \domkniecie{A}$ wtedy dla $m_1:=\phi\circ m, m_2:=m, x:=x, y:=y, \overline{x}:=\overline{x}, \overline{y}:=\overline{y}$ mamy:
$$
F(\phi\circ m;\overline{x})=F(\phi\circ m;\overline{y}) \iff F(m;\overline{x})=F(m;\overline{y}).
$$
Następujące stwierdzenie jest równoważne
\begin{equation}
\mathcal{F}(\domkniecie{\phi \circ m}(\overline{x}))=\mathcal{F}(\domkniecie{\phi \circ m}(\overline{y})).
\end{equation}
Zauważmy, że
$$
F(\phi \circ m;\overline{x})=\mathcal{F}( \domkniecie{\phi \circ m}(\overline{x})), \quad F(\phi \circ m;y)=\mathcal{F}(\domkniecie{\phi\circ m}(\overline{y})).
$$
Zatem równoważne zdanie (3.8) jest równe zdaniu
$$
F(\phi\circ m;\overline{x})=F(\phi \circ m;\overline{y}).
$$
Dzięki warunkowi porównawczej niezmienniczości to z kolei jest równoważne
$$
F(m;\overline{x})=F(m;\overline{y}).
$$
Zauważmy, że 
$$
F(m;\overline{x})=\mathcal{F}(\domkniecie{m}(\overline{x})) \quad \textrm{oraz} \quad F(m;\overline{y})=\mathcal{F}(\domkniecie{m}(\overline{y})).
$$
Zatem ostatecznie jest równoważne 
$$
\mathcal{F}(\domkniecie{m}(\overline{x}))=\mathcal{F}(\domkniecie{m}(\overline{y})).
$$
Z dowolności $\overline{x}, \overline{y} \in \domkniecie{A}$, $m_1, m_2 \in M$ udowodniliśmy punkt (2).

Udowodnimy (2) $\implies (1)$
Załóżmy, że dla dow. $m \in M$, $\phi \in \Phi$ i $\overline{x}$, $\overline{y} \in \domkniecie{A}$ o ile $\domkniecie{m}(\overline{x}), \domkniecie{
m}(\overline{y}), \domkniecie{\phi\circ m}(\overline{x})), \domkniecie{\phi\circ m}(\overline{y})) \in \mathcal{D}$ to mamy:
\begin{equation}
\mathcal{F}(\domkniecie{\phi \circ m}(\overline{x}))=\mathcal{F}(\domkniecie{\phi \circ m}(\overline{y}))) \iff \mathcal{F}(\domkniecie{m}(\overline{x}))=\mathcal{F}(\domkniecie{m}(\overline{y})).
\end{equation}
Niech $m_1,m_2 \in M, \overline{x}, \overline{y} \in \domkniecie{A}$. Z regularności NAS istnieją $m_0, \phi_1,\phi_2$ takie, że
$$ 
m_1=\phi_1\circ m_0, \quad m_2=\phi_2\circ m_0.
$$
Niech $\overline{\alpha}=\domkniecie{m_0}(\overline{x})$ wtedy ponownie z regularności NAS istnieje $\Psi$ takie, że dla każdego $\overline{\alpha_i} \in \alpha$
$$
\Psi(\phi_2(\overline{\alpha_i}))=\overline{\alpha_i}
$$
Dla $m:=\phi_2 \circ m_0$, $\phi:=\phi_1\circ \Psi$, $\overline{x}:=\overline{x}, \overline{x}:=\overline{x}$ mamy 
$$
\mathcal{F}(\domkniecie{\phi_1\circ \Psi \circ \phi_2\circ m_0}(\overline{x}))=\mathcal{F}(\domkniecie{\phi_1\circ \Psi \circ \phi_2\circ m_0}(\overline{y})) \iff \mathcal{F}(\domkniecie{\phi_2\circ m_0}(\overline{x}))=(\domkniecie{\phi_2\circ m_0}(\overline{y})).
$$
\begin{eqnarray*}
F(m_1;\overline{x})=F(m_1;\overline{y}) & \iff  & \mathcal{F}(\domkniecie{\phi_1\circ m_0}(\overline{x}))=\mathcal{F}(\domkniecie{\phi_1 \circ m_0}(\overline{y}))\\
&  \iff & \mathcal{F}(\domkniecie{\phi_1\circ\Psi\circ\phi_2\circ m_0}(\overline{x}))=\mathcal{F}(\domkniecie{\phi_1\circ\Psi\circ\phi_2\circ m_0.}(\overline{y})) 
\end{eqnarray*}
Wobec (3.9)
\begin{eqnarray*}
& \iff & \mathcal{F}(\domkniecie{\phi_2\circ m_0}(\overline{x}))=\mathcal{F}(\domkniecie{\phi_2\circ m_0}(\overline{y}))\\
& \iff & F(m_2;\overline{x})=F(m_2;\overline{y}).
\end{eqnarray*}

Z dowolności $m_1, m_2$ i $\overline{x}$, $\overline{y}$, $F$ jest porównawczo niezmiennicza.

Udowodnimy (3) $\implies$ (2)
Załóżmy, że dla dow. $\phi \in \Phi$ i $\overline{\alpha}$,$ \overline{\beta} \in \domkniecie{a}$ o ile $\overline{\alpha}$, $\overline{\beta}$, $\domkniecie{\phi}(\overline{\alpha})$, $\phi(\overline{\beta}) \in \mathcal{D}$ to mamy:
\begin{equation}
\mathcal{F}(\domkniecie{\phi}(\overline{\alpha}))=\mathcal{F}(\domkniecie{\phi}(\overline{\beta})) \iff \mathcal{F}(\overline{\alpha})=\mathcal{F}(\overline{\beta}).
\end{equation}
Niech $m\in M, \phi\in \Phi$ i $\overline{x}, \overline{y} \in \domkniecie{A}$, $\overline{\alpha}:=\domkniecie{m}(\overline{x})$, $\overline{\beta}:=\domkniecie{m}(\overline{y})$. Z (3.10)
$$
\mathcal{F}(\domkniecie{\phi\circ m}(\overline{x}))=\mathcal{F}(\domkniecie{\phi\circ m}(\overline{x})) \iff \mathcal{F}(\domkniecie{m} (\overline{x})=\mathcal{F}(\domkniecie{m}(\overline{y})).
$$
Z dowolności wyboru $m, \phi$ oraz $\overline{x}$, $\overline{y}$ udowodniliśmy punkt (2).

Udowodnimy $(2) \implies (3)$
Załóżmy, że dla dow. $m \in M$, $\phi \in \Phi$ i $\overline{x}$, $\overline{y} \in \domkniecie{A}$ o ile $\domkniecie{m}(\overline{x}), \domkniecie{
m}(\overline{y}), \domkniecie{\phi\circ m}(\overline{x})), \domkniecie{\phi\circ m}(\overline{y})) \in \mathcal{D}$ to mamy.
\begin{equation}
\mathcal{F}(\domkniecie{\phi \circ m}(\overline{x}))=\mathcal{F}(\domkniecie{\phi \circ m}(\overline{y}))) \iff \mathcal{F}(\domkniecie{m}(\overline{x}))=\mathcal{F}(\domkniecie{m}(\overline{y})).
\end{equation}
Niech $\phi \in \phi$, $\overline{\alpha}, \overline{\beta} \in \domkniecie{a}$ oraz $\overline{\alpha}$, $\overline{\beta}$, $\phi(\overline{\alpha})$, $\phi(\overline{\beta}) \in \mathcal{D}$ i $\domkniecie{m}(\overline{x}):=\overline{\alpha}$,  $\domkniecie{m}(\overline{y}):=\overline{\beta}$. 
Rozumujemy równoważnie
$$
\mathcal{F}(\domkniecie{\phi}(\overline{\alpha}))=\mathcal{F}(\domkniecie{\phi}(\overline{\beta}).
$$
Niech $m \in M$, $\overline{x}$, $\overline{y} \in \domkniecie{A}$ takie, że
$$
\overline{\alpha}=\domkniecie{m}(\overline{x}), \quad \overline{\beta}=\domkniecie{m}(\overline{y}).
$$
Wtedy z pierwszej równoważności
$$
\mathcal{F}(\domkniecie{\phi \circ m}(\overline{x})=\mathcal{F}(\domkniecie{\phi \circ m}(\overline{y}).
$$
Z (3.11)
$$
\mathcal{F}(\domkniecie{m}(\overline{x}))=\mathcal{F}(\domkniecie{m}(\overline{y})), 
$$
$$
\mathcal{F}(\overline{\alpha})=\mathcal{F}(\overline{\beta}).
$$
Z dowolności $\phi$, $\overline{\alpha}$, $\overline{\beta}$ udowodniliśmy punkt (3).
\end{proof}


\begin{tw}[{\citep[Tw. 2]{adams1965theory}}]
Niech $U=\tuple{A,M,a,\phi}$ będzie regularnym NAS, $\mathcal{F}:\mathcal{D}\to \domkniecie{\mathbb{R}}$, oraz niech $F$ będzie uogólnioną operacją na pomiarach odpowiadającą $\mathcal{F}$. Wtedy:
\begin{enumerate}
\item
$F$ jest bezwzględnie niezmiennicza w stosunku do $U$ 
$$
\forall_{\phi \in \Phi} \forall_{\overline{\alpha} \in \domkniecie{a}} \quad \overline{\alpha}, \domkniecie{\phi}(\overline{\alpha}) \in \mathcal{D} \implies \mathcal{F}(\domkniecie{\phi}(\overline{\alpha}))=\mathcal{F}(\overline{\alpha}).
$$
\item
Jeżeli dla wszystkich $\overline{\alpha} \in \domkniecie{a}$ takich, że $\overline{\alpha} \in \mathcal{D}$, $\mathcal{F}(\alpha)\in a$, wtedy $F$ jest odnośnikowo niezmiennicza w stosunku do $U$ gddy
\begin{equation*}
\forall_{\phi \in \Phi} \forall_{\overline{\alpha} \in \domkniecie{a}} \quad \overline{\alpha}, \domkniecie{\phi}(\overline{\alpha}) \in \mathcal{D} \implies \mathcal{F}(\domkniecie{\phi}(\overline{\alpha}))=\domkniecie{\phi}(\mathcal{F}(\overline{\alpha})).
\end{equation*}
\item
$F$ jest porównawczo niezmiennicza w stosunku do $U$ gddy dla wszystkich $\phi \in \Phi$ istnieje $\Psi_{\phi}$ które jest różnowartościowe takie, że
$$
 \forall_{\overline{\alpha} \in \domkniecie{a}} \quad \overline{\alpha}, \domkniecie{\phi}(\overline{\alpha}) \in \mathcal{D} \implies \mathcal{F}(\domkniecie{\phi}(\overline{\alpha}))=\Psi_{\phi}(\mathcal{F}(\overline{\alpha})).
$$
\end{enumerate}
\end{tw}
\begin{proof} 
\begin{enumerate}
\item
Dowód (1) wynika wprost z twierdzenia 3.13
\item
Udowodnimy warunek dostateczny. Załóżmy, że 
\begin{equation}
\forall_{\phi \in \Phi} \forall_{\overline{\alpha} \in \domkniecie{a}} \quad \overline{\alpha}, \domkniecie{\phi}(\overline{\alpha}) \in \mathcal{D} \implies \mathcal{F}(\domkniecie{\phi}(\overline{\alpha}))=\domkniecie{\phi}(\mathcal{F}(\overline{\alpha})).
\end{equation}
Niech $m_1, m_2 \in M$ $x \in A$ i $\overline{x} \in \domkniecie{A}$ i $\domkniecie{m_1}(\overline{x}), \domkniecie{m_2}(\overline{x}) \in \mathcal{D}$
$$
m_1(x)=F(m_1;\overline{x})=F(\domkniecie{m_1}(\overline{x})).
$$
Z regularności NAS istnieje $m_0, \phi_1 , \phi_2$ takie, że 
$$ 
m_1=\phi_1\circ m_0, \quad m_2=\phi_2\circ m_0.
$$
Niech $\alpha=m_0(x)$, $\overline{\alpha}=\domkniecie{m_0}(\overline{x})$, $\alpha=(\alpha_1,\dots,\alpha_n)$.

Dalej z regularności $U$ istnieje $\Psi:a \to a$
$$
 \Psi(\phi_1(\overline{\alpha_i}))=\overline{\alpha_i},
$$
$$
 \Psi(\phi_1(\alpha))=\alpha.
$$
\begin{eqnarray*}
m_1(x) & = & \mathcal{F}(\domkniecie{m_1}(\overline{x})),\\ 
\phi_1 \circ m_0(x)& = & \mathcal{F}(\domkniecie{\phi_1 \circ m_0}(\overline{x}))\quad /\Psi(\cdot),\\ 
\Psi \circ \phi_1 \circ m_0(x)&= & \Psi(\mathcal{F}(\domkniecie{\psi_1 \circ m_0}(\overline{x})))\quad /\phi_2(\cdot),\\
m_2(x)&=&\phi_2(\Psi(\mathcal{F}(\domkniecie{\phi_1 \circ m_0}(x))).
\end{eqnarray*}

Dla $\phi:=\phi_2\circ \Psi,$ $\overline{\alpha}:=\domkniecie{m_1}(\overline{x})$ z (3.12)
\begin{eqnarray*}
m_2(x) & = & \mathcal{F}(\domkniecie{\phi_2 \circ \Psi \circ \phi_1 \circ m_0})(\overline{x})),\\
m_2(x) & = & \mathcal{F}(\domkniecie{m_2}(\overline{x})).
\end{eqnarray*}
Z uwagi na symetrię oznaczeń implikacja w drugą stronę jest oczywista. Z dowolności $m_1, m_2$, $x$ i $\overline{x}$, F jest odnośnikowo niezmiennicza.

Udowodnimy warunek konieczny. Załóżmy, że dla dow. $\phi \in \Phi$, $\alpha \in a$ i $\overline{\alpha} \in \domkniecie{a}$ o ile $\overline{\alpha}, \domkniecie{\phi}(\overline{\alpha}) \in \mathcal{D}$ to mamy: 
\begin{equation}
\phi(\alpha))=\mathcal{F}(\domkniecie{\phi}(\overline{\alpha})) \iff \alpha=\mathcal{F}(\overline{\alpha}).
\end{equation}
Niech $\phi \in \Phi$, $\overline{\alpha} \in \domkniecie{a}$ i $\overline{\alpha}$, $\phi(\overline{\alpha}) \in \mathcal{D}$. Niech $\alpha=\mathcal{F}(\overline{\alpha})$ z (3.13) mamy 
$$
\phi(\alpha)=\mathcal{F}(\domkniecie{\phi}(\overline{\alpha})),
$$
 jednocześnie $\alpha=\mathcal{F}(\overline{\alpha}))$ zatem
$$
\mathcal{F}(\domkniecie{\phi}(\overline{\alpha}))=\phi(\alpha)=\phi(\mathcal{F}(\overline{\alpha})).
$$
Z dowolności $\phi$ i $\alpha, \overline{\alpha}$ udowodniliśmy punkt (2).
\item
Udowodnimy warunek dostateczny. Załóżmy, że
\begin{equation*}
\forall_{\phi \in \Phi}  \exists_{\psi_{\phi}\in \Phi} \forall_{\overline{\alpha} \in \domkniecie{a}} \quad \overline{\alpha}, \domkniecie{\phi}(\overline{\alpha}) \in \mathcal{D} \implies \mathcal{F}(\phi(\overline{\alpha}))=\Psi_{\phi}(\mathcal{F}(\overline{\alpha})).
\end{equation*}
Niech $\phi \in \Phi$. Wtedy istnieje 
\begin{equation}
\Psi_{\phi}: \forall_{\overline{\alpha} \in \domkniecie{a}} \mathcal{F}(\domkniecie{\phi}(\overline{\alpha}))=\Psi_{\phi}(\mathcal{F}(\overline{\alpha})).
\end{equation}

Niech $\overline{\alpha}, \overline{\beta} \in \domkniecie{a}$. Wtedy z (3.14)
\begin{eqnarray*}
\mathcal{F}(\domkniecie{\phi}(\overline{\alpha}))&=&\Psi_{\phi}(\mathcal{F}(\overline{\alpha})),\\
\mathcal{F}(\domkniecie{\phi}(\overline{\beta}))&=&\Psi_{\phi}(\mathcal{F}(\overline{\beta})).
\end{eqnarray*}
Równoważnie
\begin{eqnarray*}
\mathcal{F}(\domkniecie{\phi}(\overline{\alpha}))&=&\mathcal{F}(\domkniecie{\phi}(\overline{\beta})),\\
\Psi_{\phi}(\mathcal{F}(\overline{\alpha}))&=&\Psi_{\phi}(\mathcal{F}(\overline{\beta})).
\end{eqnarray*}
Ponieważ $\Psi_{\phi} \in \Phi$ więc jest różnowartościowa. Zatem mamy
$$
\mathcal{F}(\overline{\alpha})=\mathcal{F}(\overline{\beta}).
$$
Udowodnimy warunek konieczny. Załóżmy, że dla dow $\phi \in \Phi$, $\overline{\alpha}$, $\overline{\beta} \in \domkniecie{a}$ o ile $\overline{\alpha}, \overline{\beta}, \phi(\overline{\alpha}), \domkniecie{\phi}(\overline{\beta}) \in \mathcal{D}$ to mamy.
\begin{equation}
\mathcal{F}(\domkniecie{\phi}(\overline{\alpha}))=\mathcal{F}(\domkniecie{\phi}(\overline{\beta})) \iff \mathcal{F}(\overline{\alpha})=\mathcal{F}(\overline{\beta}).
\end{equation}
Niech $\phi \in \Phi$. Niech $\Psi_{\phi}$ będzie taka, że $\Psi_{\phi}: z\longmapsto \mathcal{F}(\domkniecie{\phi}(\overline{\alpha}))$, gdzie $\overline{\alpha} \in \domkniecie{a} \land z=\mathcal{F}(\overline{\alpha})$
Pokażemy, że definicja jest poprawna. Aby tak było wartość funkcji nie może zależeć od wyboru $\overline{\alpha}$. 
Niech $\overline{\alpha}$, $\overline{\beta} \in \domkniecie{a} \land \mathcal{F}(\overline{\alpha})=z=\mathcal{F}(\overline{\beta})$. Z (3.15) mamy natychmiast, że $\mathcal{F}(\domkniecie{\phi}(\overline{\alpha}))=\mathcal{F}(\domkniecie{\phi}(\overline{\beta})$. To oznacza jednak, że $\Psi(\mathcal{F}(\overline{\alpha}))=\Psi(\mathcal{F}(\overline{\beta}))$. Pozostaje pokazać, że $\Psi_{\phi}$ jest różnowartościowa. Niech $z_1, z_2 \in \mathcal{D}(\Psi_{\phi})$. Zatem istnieją reprezentanci $\overline{\alpha}$, $\overline{\beta}$, że $\mathcal{F}(\overline{\alpha})=z_1$, $\mathcal{F}(\overline{\beta})=z_2$.
\begin{eqnarray*}
\Psi_{\phi}(z_1)&=&\Psi_{\phi}(z_2),\\
\Psi_{\phi}(\mathcal{F}(\overline{\alpha}))&=&\Psi_{\phi}(\mathcal{F}(\overline{\beta})).
\end{eqnarray*}
Z (3.15)
$$
\mathcal{F}(\domkniecie{\phi}(\overline{\alpha}))=\mathcal{F}(\domkniecie{\phi}(\overline{\beta})) \implies \mathcal{F}(\overline{\alpha})=\mathcal{F}(\overline{\beta}) \implies z_1=z_2.
$$
Z dowolności $\phi$, $\overline{\alpha}$, $\overline{\beta}$ udowodniliśmy implikacje.
\end{enumerate}
\end{proof}
Powyższe twierdzenie pokazuje, że warunki bezwzględnej, referencyjnej oraz porównawczej niezmienniczości operacji $F$ są równoważne (przynajmniej w szerokim zakresie warunków określonych w twierdzeniu) różnym rodzajom praw transformacji, które odnoszą się do wartości $\mathcal{F}(\phi(\alpha))$ i $\mathcal{F}(\alpha)$. 
\section{Twierdzenia i ich funkcje prawdy}
\begin{equation}
m(x)=Mean(\domkniecie{m}(\overline{x})).
\end{equation}

Szczególnym przypadkiem operacji statystycznej na pomiarach jest np. funkcja $F(m; x, \overline{x})$, która przyjmuje wartość „prawda” dla określonego numerical assignment $m$, obiekt $x$ i ciąg $\overline{x}$, na wypadek, gdyby te trzy spełniały wzór (3.16). Jeśli arbitralnie reprezentujemy wartość „prawda” przez „1”, a wartość „fałsz” przez „0”, to taka funkcja $F(m; x, \overline{x})$ może być precyzyjnie zdefiniowana w ten sposób. Dziedzina $F$ to zbiór uporządkowanych trójek $m; x, \overline{x}$ takich, że $m$ jest numerical assignment z pewną dziedziną $A$, $x$ jest elementem z $A$, a $\overline{x}$ jest skończonym ciągiem elementów z $A$ i dla wszystkich $m; x, \overline{x}$ z dziedziny $F$,
\begin{equation}
F(m;x,\overline{x})= \left\{ \begin{array}{lcl}
1 & \textrm{gdy} & m(x)=Mean(\domkniecie{m}(\overline{x})),\\
0 & \textrm{wpp}\\
\end{array} \right.
\end{equation}

Powyższe sugeruje, że przynajmniej dla ograniczonej klasy zdań statystycznych, takich jak (3.16), wzory te mogą być wykonane tak, aby odpowiadały funkcjom prawdy w prosty i jednolity sposób.
\begin{definition}[{\citep[Def. 6]{adams1965theory}}]
Uogólniona funkcja prawdy pomiaru nazywamy uogólnioną operacją na pomiarach, których wartości wynoszą tylko $0$ i $1$.
\end{definition}


Mając powiązane funkcje prawdy z wzorami w sposób opisany powyżej, możliwe staje się zastosowanie precyzyjnych pojęć niezmienniczości zdefiniowanych w poprzedniej sekcji do funkcji prawdy (przynajmniej do tych, które są uogólnionymi operacjami na pomiarach). Idea kryjąca się za definicją empirycznego znaczenia dla stwierdzeń (dokładniej wzorów) polega na tym, że ich prawda nie może być zmieniona przez zmianę numerical assignment. Jasne jest, że chodzi o wymóg, aby funkcja prawdy odpowiadająca wzorowi była bezwzględnie niezmiennicza w stosunku do danego systemu pomiarowego. Dlatego możemy zdefiniować:
\begin{definition}[{\citep[Def. 7]{adams1965theory}}]
Wyrażenie $W(F)$ jest empirycznie znaczące względem $U$ gdy jego funkcje prawdy są niezmiennicze bezwzględnie.
\end{definition}
Oczywiście definicja ta jest nieprecyzyjna z powodu nieprecyzyjności w warunkach wzoru i wzoru funkcji prawdy, który celowo nie został precyzyjnie zdefiniowany. 
\begin{definition}[{\citep[Def. 8]{adams1965theory}}]
$F$ jest względnie odpowiednie w stosunku do wyrażenia $W$ oraz $U$ (NAS) wtedy i tylko wtedy, gdy $W(F)$ jest empirycznie znaczące względem $U$.
\end{definition}

Teraz pokazujemy każdy z trzech specjalnych rodzajów niezmienniczości jako specjalne przypadki niezmienniczości względem wzoru lub klasy wzorów. Poszczególne wzory, o których mowa, są tym, co nazywamy bezwzględnym, referencyjnym i wzorami porównawczymi. Są one definiowane w sposób okrężny: najpierw określamy, jakie muszą być funkcje prawdy tych wzorów, a następnie definiujemy wzór, który będzie tego rodzaju, na wypadek, gdyby miał jedną z nich jako swoją funkcję prawdy. 
\begin{definition}[{\citep[Def. 9]{adams1965theory}}]
Niech $F$ będzie uogólnioną operacją na pomiarach z dziedziną $\mathcal{D}$. Wtedy:
\begin{enumerate}
\item
Dla wszystkich liczb rzeczywistych $\alpha, ABS_{F,\alpha}$ jest funkcją z dziedzina $\mathcal{D}$ taką, że dla wszystkich $m;\overline{x} \in \mathcal{D}$,
\begin{equation*}
ABS_{F,\alpha}(m;\overline{x})= \left\{ \begin{array}{lcl}
1 & \textrm{gdy} & F(m;\overline{x})=\alpha,\\
0 & \textrm{wpp}\\
\end{array} \right.
\end{equation*}
\item
$REF_{F}$ jest funkcją z dziedziną $\mathcal{D'}$ składającą się ze wszystkich uporządkowanych trójek $m;x,\overline{x}$, gdzie $m;\overline{x}$ jest w $\mathcal{D}$ i $x$ jest w dziedzinie $m$, taka że
\begin{equation*}
REF_{F}(m;x,\overline{x})= \left\{ \begin{array}{lcl}
1 & \textrm{gdy} & m(x)=F(m;\overline{x}),\\
0 & \textrm{wpp}\\
\end{array} \right.
\end{equation*}
\item
$COMP_{F}$ jest funkcją z dziedziną $\mathcal{D''}$ składająca się ze wszystkich uporządkowanych trójek $m;\overline{x},\overline{y},$ gdzie $m;\overline{x}$, $m;\overline{y} \in \mathcal{D}$ taka że
\begin{equation*}
COMP_{F}(m;\overline{x},\overline{y})= \left\{ \begin{array}{lcl}
1 & \textrm{gdy} & F(m;\overline{x})=F(m;\overline{y}),\\
0 & \textrm{wpp}
\end{array} \right.
\end{equation*}
\item
Rozważane są tylko takie wyrażenia które wykorzystują pewne funkcje $ABS_{F,\alpha}$, $REF_{F}$ lub $COMP_{F}$ jako swoje funkcje prawdy.
\end{enumerate}
\end{definition}
\begin{przyklad}[Funkcji ABS]
Rozważmy próbę losową $\overline{x}$. Niech $F$ zwraca graniczny poziom istotności z testu istotności dla $\overline{x}$ przy pomiarze $m$. Jeżeli $ F(m;\overline{x})=\alpha$ funkcja $ABS_{\alpha}$ zwróci 1 oraz w przeciwnym przypadku 0.
\end{przyklad}
\begin{przyklad}[Funkcji REF]
Osobnik $x$ ma wzrost równy średniej z populacji $\overline{x}$ wtedy funkcja $REF$ zwraca 1 w przeciwnym przypadku 0.
\end{przyklad}
\begin{przyklad}[Funkcji COMP]
Średnia wzrostu populacji $\overline{x}$ i $\overline{y}$ są równe wtedy funkcja $COMP$ zwraca 1 w przeciwnym przypadku 0.
\end{przyklad}

\begin{tw} [{\citep[Tw. 3]{adams1965theory}}]
Niech $F$ będzie uogólnioną operacją na pomiarach taka, że dla wszystkich rzeczywistych $\alpha$, $ABS_{F,\alpha}$ oraz $REF_{F}$ i $COMP_{F}$ są także uogólnionymi operacjami na pomiarach, oraz niech $U$ będzie NAS. Wtedy:
\begin{enumerate}
\item
$F$ jest bezwzględnie niezmiennicza w stosunku do $U$ gddy dla dow. $\alpha$, jest względnie odpowiednia w stosunku do $ABS_{F,\alpha}$ i $U$.
\item
$F$ jest odnośnikowo niezmiennicza w stosunku do $U$ gddy, jest względnie odpowiednia w stosunku do $REF_{F}$ i $U$.  
\item
$F$ jest porównawczo niezmiennicza w stosunku do $U$ gddy, jest względnie odpowiednia w stosunku do $COMP_{F}$ i $U$.    
\end{enumerate}
\end{tw}
\begin{proof}
\begin{enumerate}
\item
Udowodnimy warunek konieczny.
Niech $F$ będzie bezwzględnie niezmiennicza w stosunku do $U$ 
\begin{equation}
\forall_{m_{1}, m_{2} \in M}  \forall_{\overline{x} \in \domkniecie{A}} \quad \domkniecie{m_1}(\overline{x}), \domkniecie{m_2}(\overline{x})\in \mathcal{D} \implies F(m_{1};\overline{x})=F(m_{2};\overline{x}).
\end{equation}
Niech $\alpha \in \mathbb{R}$. Pokażemy, że $ABS_{F,\alpha}$ jest bezwzględnie niezmiennicza. Niech $m_1$, $m_2 \in M$ oraz niech $\overline{x} \in \domkniecie{A}$ 
\begin{enumerate}[I.]
\item
Jeśli $ABS_{F,\alpha}(m_1;\overline{x})=1$ to 
$$
F(m_1;\overline{x})=\alpha.
$$ 
Z (3.18) 
$$
F(m_2;\overline{x})=\alpha.
$$
Stąd 
$$
ABS_{F,\alpha}(m_2;\overline{x})=1.
$$
\item
Jeśli $ABS_{F,\alpha}(m_1;\overline{x})=0$ to 
$$
F(m_1;\overline{x})\ne \alpha.
$$
Z (3.18) 
$$
F(m_2;\overline{x})\ne\alpha.
$$
Stąd 
$$
ABS_{F,\alpha}(m_2;\overline{x})=0.
$$
\end{enumerate}
Zatem 
$$
ABS_{F,\alpha}(m_1;\overline{x})=ABS_{F,\alpha}(m_2;\overline{x}).
$$
Z dowolności $m_1$, $m_2$, $\overline{x}$ oraz $\alpha$ pokazaliśmy, że $ABS_{F,\alpha}$ jest bezwzględnie niezmiennicza.

Udowodnimy warunek dostateczny.Rozumujemy nie wprost. Przypuśćmy, że $F$ nie jest bezwzględnie niezmiennicza. Zatem istnieją $m_1$, $m_2 \in M$ oraz $\overline{x} \in \domkniecie{A}$ takie, że
$$
F(m_1;\overline{x})\ne F(m_2;\overline{x}).
$$
Niech $\alpha=F(m_1;\overline{x})$. Rozważmy wyrażenie $ABS_{F,\alpha}$  z założenia jest ono bezwzględnie niezmiennicze wtedy $ABS_{F,\alpha}(m_1;\overline{x})=1$. Skoro jest bezwzględnie niezmiennicza  to $ABS_{F,\alpha}(m_1;\overline{x})=ABS_{F,\alpha}(m_2;\overline{x})=1$. Z definicji $ABS_{F,\alpha}(m_2;\overline{x})$ mamy, że $F(m_2;\overline{x})=\alpha$, a $\alpha=F(m_1;\overline{x})$. Sprzeczność, jest ona efektem takiego przypuszczenia, że $F(m_1;\overline{x})\ne F(m_2;\overline{x})$, stad teza.
\item
Udowodnimy warunek konieczny. Niech $F$ będzie odnośnikowo niezmiennicza w stosunku do $U$. Dla dow. $m_{1}$, $m_{2} \in M$ $x \in A$ i  $\overline{x} \in \domkniecie{A}$ o ile  $\domkniecie{m_1}(\overline{x})$, $\domkniecie{m_2}(\overline{x})\in \mathcal{D}$ to mamy.
\begin{equation}
 m_{1}(x)=F(m_{1};\overline{x}) \iff m_{2}(x)=F(m_{2};\overline{x}).
\end{equation}
Pokażemy, że $REF_{F}$ jest bezwzględnie niezmiennicza. Niech $m_1$, $m_2 \in M$ oraz niech $\overline{x} \in \domkniecie{A}$ 
\begin{enumerate}[I.]
\item
Jeśli $REF_{F}(m_1;x,\overline{x})=1$ to z definicji
$$
m_1(x)=F(m_1;\overline{x}).
$$
Z (3.19) 
$$
m_2(x)=F(m_2;\overline{x}).
$$
Stąd 
$$
REF_{F}(m_2;x,\overline{x})=1.
$$
\item
Jeśli $REF_{F}(m_1;x,\overline{x})=0$ to z definicji 
$$
m_1(x)\ne F(m_1;\overline{x}).
$$
Z (3.19)
$$
m_2(x)\ne F(m_2;\overline{x}).
$$
Stąd
$$
REF_{F}(m_2;x,\overline{x})=0.
$$
\end{enumerate}
Zatem 
$$
REF_{F}(m_1;x,\overline{x})=REF_{F}(m_2;x,\overline{x}).
$$
Z dowolności $m_1$, $m_2$, $\overline{x}$, $x$ oraz $\alpha$ pokazaliśmy, że $REF_{F}$ jest bezwzględnie niezmiennicza.

Udowodnimy warunek dostateczny.Rozumujemy nie wprost. Przypuśćmy, że $F$ nie jest bezwzględnie niezmiennicza. Zatem istnieją $m_1$, $m_2 \in M$ oraz$x\in A$ $\overline{x} \in \domkniecie{A}$ takie, że
$$
m_{1}(x)=F(m_{1};\overline{x}) \notiff m_{2}(x)=F(m_{2};\overline{x}).
$$
\begin{enumerate}[I.]
\item
Jeśli $m_1(x)=F(m_1;\overline{x})$ wtedy 
$$
REF_{F}(m_1;x,\overline{x})=1.
$$
Z bezwzględnej niezmienniczości
 $$
 REF_{F}(m_2;x,\overline{x})=1.
 $$
Wtedy
$$
m_2(x)=F(m_2;\overline{x}).
$$
Sprzeczność.
\item
Jeśli $m_1(x)\ne F(m_1;\overline{x})$ wtedy 
$$
REF_{F}(m_1;x,\overline{x})=0.
$$
Z bezwzględnej niezmienniczości 
$$
REF_{F}(m_2;x,\overline{x})=0.
$$ 
Wtedy
$$
m_2(x)\ne F(m_2;\overline{x}).
$$
 Sprzeczność.
\end{enumerate}
Skoro wszystkie przypadki skończyły się uzyskaniem sprzeczności to nasze przypuszczenie musi być fałszywe.
\item 
Udowodnimy warunek konieczny. Niech $F$ będzie porównawczo niezmiennicza w stosunku do $U$ 
dla dow. $m_1$, $m_2 \in M$ i $\overline{x}$, $\overline{y} \in \domkniecie{A}$ o ile $\domkniecie{m_1}(\overline{x}), \domkniecie{ m_2}(\overline{x}), \domkniecie{m_1}(\overline{y}), \domkniecie{m_2}(\overline{y}) \in \mathcal{D}$ to mamy:
\begin{equation}
F(m_{1};\overline{x})=F(m_{1};\overline{y}) \iff F(m_{2};\overline{x})=F(m_{2};\overline{y}).
\end{equation}
Pokażemy, że $COMP_{F}$ jest bezwzględnie niezmiennicza, to znaczy
\begin{equation}
\forall_{m_1, m_2 \in M} \forall_{\overline{x}, \overline{y}\in \domkniecie{A}} \quad COMP_{F}(m_1;\overline{x},\overline{y})=COMP_{F}(m_2;\overline{x},\overline{y}).
\end{equation}
 Niech $m_1$, $m_2 \in M$ oraz niech $\overline{x}$, $\overline{y} \in \domkniecie{A}$. Wtedy
\begin{enumerate}[I.]
\item
Jeśli $COMP_{F}(m_1;\overline{x},\overline{y})=1$ to
$$
F(m_1;\overline{x})=F(m_1;\overline{y}).
$$
To z (3.20) 
$$
F(m_2;\overline{x})=F(m_2;\overline{y}).
$$
Stąd 

$$
COMP_{F}(m_2;\overline{x},\overline{y})=1.
$$
\item
Jeśli $COMP_{F}(m_1;\overline{x},\overline{y})=0$ to 
$$
F(m_1;\overline{x})\ne F(m_1;\overline{y}).
$$
To z (3.20) 
$$
F(m_2;\overline{x})\ne F(m_2;\overline{y}).
$$
Stąd 
$$
COMP_{F}(m_2;\overline{x},\overline{y})=0.
$$
\end{enumerate}
Zatem 
$$
COMP_{F}(m_1;\overline{x},\overline{y})=COMP_{F}(m_2;\overline{x},\overline{y}).
$$
Z dowolności $m_1$, $m_2$, $\overline{x}$, $\overline{y}$ pokazaliśmy, że $COMP_{F}$ jest bezwzględnie niezmiennicza.

Udowodnimy warunek dostateczny. Rozumujemy nie wprost. Przypuśćmy, że $F$ nie jest bezwzględnie niezmiennicza. Zatem istnieją $m_1$, $m_2 \in M$ oraz$x\in A$ $\overline{x}, \overline{y} \in \domkniecie{A}$ takie, że
$$
F(m_{1};\overline{x})=F(m_{1};\overline{y}) \notiff F(m_{2};\overline{x})=F(m_{2};\overline{y}).
$$
\begin{enumerate}[I.]
\item
Jeśli $F(m_1;\overline{x})=F(m_1;\overline{y})$ wtedy
$$
COMP_{F}(m_1;\overline{x},\overline{y})=1.
$$
Z (3.21)
$$
COMP_{F}(m_2;\overline{x},\overline{y})=1.
$$
Z def $COMP_F$
$$
F(m_2;\overline{x})=F(m_2;\overline{y}).
$$ 
Sprzeczność.
\item
Jeśli $F(m_1;\overline{x})\ne F(m_1;\overline{y})$ wtedy 
$$
COMP_{F}(m_1;\overline{x},\overline{y})=0.
$$
Z (3.21) 
$$
COMP_{F}(m_2;\overline{x},\overline{y})=0.
$$
Z def $COMP_F$
$$
F(m_2;\overline{x})\ne F(m_2;\overline{y}).
$$ 
Sprzeczność.
\end{enumerate}
Skoro wszystkie przypadki skończyły się uzyskaniem sprzeczności to nasze przypuszczenie musi być fałszywe.
\end{enumerate}
\end{proof}

\chapter{Podsumowania}
W rozdziale 3. Zostały wprowadzone definicje NAS oraz NAS regularnego, który ma mocniejsze założenia od zwykłego ale jednak słabsze niż założenia grup algebraicznych. Problem ten został również zilustrowany na przykładach. Omówione również zostały typy skal. Następnie zostały wprowadzone definicje domknięcia zbioru skończonymi ciągami, rozszerzenia funkcji na skończone ciągi, uogólnionych funkcji rzeczywistych oraz uogólnionego działania na pomiarach które pozwoliły zdefiniować nam niezmienniczość dla uogólnionych działań na pomiarach. Są trzy rodzaje niezmienniczości; bezwzględna, odnośnikowa i porównawcza. Dla każdego rodzaju niezmienniczości zostały sformułowane warunki równoważne oraz udowodnione.  Zostało wprowadzone również twierdzenie 3.14, początkowe próby dowodzenia punktu trzeciego nie powiodły się więc konieczne było dodatkowe  założenie że funkcja $\Psi_{\phi}$ jest różnowartościowa. Bez tego warunku dowód nie był możliwy do wykonania.
\bibliographystyle{plain}
\bibliography{bibliografia}
\end{document}